\section{Conclusiones Generales}

Este trabajo fin de grado ha cumplido con el objetivo general de materializar un \gls{mvp} funcional, cerrando con éxito la fase \textit{Construir} del ciclo expuesto en el libro 
\textit{The Lean Startup} \cite{ries2011lean} que ha servido como referencia y marco metodológico durante todo el desarrollo del proyecto. Además, a través del desarrollo de PANOT, 
se ha demostrado que la Inteligencia Relacional puede ser asistida tecnológicamente mediante un motor multi-agente sin sacrificar simplicidad de uso.

\noindent Respecto a los objetivos específicos:

\begin{itemize}
    \item Se ha fundamentado el cambio de paradigma del contacto tabular al grafo de conocimiento (\hyperref[o1]{O1}).
    \item Se ha implementado un motor multiagéntico (\hyperref[o2]{O2}) y una interfaz \textit{voice-first} (\hyperref[o4]{O4}) que reducen drásticamente la fricción de entrada de datos.
    \item La arquitectura \textit{local-first} (\hyperref[o3]{O3}) y los mecanismos de seguridad (\hyperref[o6]{O6}) garantizan una herramienta eficiente y privada.
    \item El sistema ha sido instrumentado y desplegado con éxito (\hyperref[o5]{O5} y \hyperref[o7]{O7}), estando ya disponible globalmente y listo para iniciar la fase de \textit{Medir}.
\end{itemize}
