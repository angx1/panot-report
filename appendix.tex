\appendix

\chapter{Evidencias de Verificación de Requisitos No Funcionales}

\section{\hyperref[req:RNF-SEG-001]{RNF-SEG-001} - Seguridad}
\label{appendix:seguridad}

\begin{figure}[H]
    \centering
    \includegraphics[width=1\textwidth]{figures/rls-graph.png}
    \caption{Política RLS para las tablas \texttt{semantic\_edges} y \texttt{semantic\_nodes}.}
    \label{fig:rls-graph}
\end{figure}

\begin{figure}[H]
    \centering
    \includegraphics[width=1\textwidth]{figures/rls-interactions.png}
    \caption{Política RLS para la tabla \texttt{interactions}.}
    \label{fig:rls-interactions}
\end{figure}


\begin{figure}[H]
    \centering
    \includegraphics[width=1\textwidth]{figures/rls-profiles.png}
    \caption{Política RLS para la tabla \texttt{profiles}.}
    \label{fig:rls-profiles}
\end{figure}

\begin{figure}[H]
    \centering
    \includegraphics[width=1\textwidth]{figures/rls-contacts.png}
    \caption{Política RLS para la tabla \texttt{contacts}.}
    \label{fig:rls-contacts}
\end{figure}


\begin{figure}[H]
    \centering
    \includegraphics[width=1\textwidth]{figures/rls-workers.png}
    \caption{Política RLS para la tabla \texttt{workers}.}
    \label{fig:rls-workers}
\end{figure}

\begin{figure}[H]
    \centering
    \includegraphics[width=1\textwidth]{figures/rls-process-queue.png}
    \caption{Política RLS para la tabla \texttt{process\_queue}.}
    \label{fig:rls-process-queue}
\end{figure}

\begin{figure}[H]
    \centering
    \includegraphics[width=1\textwidth]{figures/rls-tickets.png}
    \caption{Política RLS para la tabla \texttt{feedback\_tickets}.}
    \label{fig:rls-tickets}
\end{figure}


\newpage
\section{\hyperref[req:RNF-REND-002]{RNF-REND-002} y \hyperref[req:RNF-EFI-003]{RNF-EFI-003} - Rendimiento y Eficiencia}
\label{sec:rendimiento-eficiencia}

Para validar los requisitos de rendimiento y eficiencia económica del sistema multi-agente de PANOT, se ha realizado una evaluación de carga consistente en 27 ejecuciones consecutivas del 
sistema ({\footnotesize \texttt{relational\_agent}}) utilizando transcripciones reales de longitud media (aprox. 300 palabras). Los datos de los resultados han sido extraídos de la plataforma de 
observabilidad \textit{LangSmith}.

\vspace{.5cm}
\begin{figure}[H]
    \centering
    \includegraphics[width=0.7\textwidth]{figures/lista-trazas.png}
    \caption{Lista de trazas de ejecución en LangSmith.}
    \label{fig:lista-trazas}
\end{figure}


\subsection{Análisis de Latencia - \hyperref[req:RNF-REND-002]{RNF-REND-002}}
\label{sec:latencia}

Dada la naturaleza estocástica de los modelos de lenguaje, el tiempo de respuesta ($T$) de una solicitud presentan una alta varianza debida a la carga del modelo, el tamaño de la ventana de contexto y la profundidad de la inferencia. 
Para evaluar el cumplimiento del requisito \hyperref[req:RNF-REND-002]{RNF-REND-002}, se analizan las siguientes métricas estadísticas sobre la muestra de las $n=27$ trazas:

\begin{enumerate}
    \item \textbf{Latencia End-to-End (E2E) ($T_{e2e}$):} Define el tiempo total desde el inicio del proceso hasta la respuesta final. Para cada observación $i$, se calcula como:
    \begin{equation}
        T_{e2e,i} = t_{final,i} - t_{inicio,i}
    \end{equation}

    \item \textbf{Teimpo de Respuesta Medio ($\bar{T}$):} La media aritmética de los tiempos de respuesta. Esta métrica es el indicador principal de rendimiento esperado bajo una carga constante.
    \begin{equation}
        \bar{T} = \frac{1}{n} \sum_{i=1}^{n} T_{e2e,i}
    \end{equation}

    \item \textbf{Mediana ($M$):} Representa el valor central de la muestra. En sistemas agenticos, se utiliza para neutralizar el sesgo de los valores atípicos (\textit{outliers}) provocados por la 
    latencia variable de las \acrshort{api} de terceros. Para una muestra ordenada $x_{(1)} \leq x_{(2)} \leq \ldots \leq x_{(n)}$:
    \begin{equation}
        M = \begin{cases}
            x_{((n+1)/2)} & \text{si } n \text{ es impar (nuestro caso: $n=27$)} \\[0.5em]
            \dfrac{x_{(n/2)} + x_{(n/2+1)}}{2} & \text{si } n \text{ es par}
        \end{cases}
    \end{equation}

\end{enumerate}

\subsubsection{Resultados}

\begin{figure}[H]
    \centering
    \includegraphics[width=0.9\textwidth]{figures/latencia-trazas.png}
    \caption{Gráfica de latencia por traza en LangSmith. Listado de valores de latencia ($T_{e2e}$): 28.89; 37.34; 34.37; 46.75; 34.03; 40.45; 30.48; 29.06; 30.60; 31.19; 34.84; 31.96; 35.06; 32.99; 35.42; 37.17; 28.09; 45.99; 32.48; 29.69; 34.50; 7.75; 32.07; 32.42; 31.39; 34.37; 37.60}
    \label{fig:latencia-trazas}
\end{figure}

Tras el procesamiento de la muestra (Fig. \ref{fig:lista-trazas}), los resultados arrojan una \textbf{Latencia Media ($\bar{T}$)} de \textbf{33,22 segundos} y una \textbf{Mediana ($M$)} de 
\textbf{32,99 segundos}. Ambos valores se sitúan por debajo del \textit{threshold} de \textbf{35 segundos} establecido como indicador de éxito.

\vspace{1cm}
\subsection{Análisis de Eficiencia Económica - \hyperref[req:RNF-EFI-003]{RNF-EFI-003}}
\label{sec:eficiencia-economica}

La eficiencia económica del sistema se evalúa en función del coste operativo directo por cada ejecución del sistema agéntico. Este coste está determinado por el consumo de 
recursos computacionales del procesamiento de lenguaje natural, el cual se tarifica mediante el uso de \textit{tokens} de entrada y salida. Para validar el cumplimiento 
del requisito \hyperref[req:RNF-EFI-003]{RNF-EFI-003}, se analizan las siguientes métricas sobre la muestra de las $n=27$ trazas:

\begin{enumerate}
    \item \textbf{Coste por Ejecución ($C_i$):} Representa el gasto monetario de una traza individual, derivado del volumen de información procesada ($I$: \textit{tokens} de 
    entrada, $O$: \textit{tokens} de salida) y el coste por \textit{token} del modelo ($\tau$):
    \begin{equation}
        C_i = (I_i \cdot \tau_{in}) + (O_i \cdot \tau_{out})
    \end{equation}

    \item \textbf{Coste Medio ($\bar{C}$):} El promedio aritmético del coste de todas las ejecuciones de la muestra, utilizado para proyectar la viabilidad económica a largo plazo:
    \begin{equation}
        \bar{C} = \frac{1}{n} \sum_{i=1}^{n} C_i
    \end{equation}
\end{enumerate}

\subsubsection{Resultados}

\begin{figure}[H]
    \centering
    \includegraphics[width=0.9\textwidth]{figures/coste-trazas.png}
    \caption{Gráfica de coste por traza en LangSmith. Listado de valores de coste ($C_i$) en USD: 0.0044; 0.0042; 0.0052; 0.0046; 0.0049; 0.0056; 0.0044; 0.0052; 0.0048; 0.0047; 0.0045; 0.0045; 0.0057; 0.0047; 0.0050; 0.0055; 0.0051; 0.0080; 0.0054; 0.0047; 0.0054; 0.0012; 0.0045; 0.0052; 0.0051; 0.0060; 0.0060.}
    \label{fig:coste-trazas}
\end{figure}

Tras analizar los datos de facturación de las $n=27$ trazas (Fig. \ref{fig:coste-trazas}), el \textbf{Coste Medio ($\bar{C}$)} resultante es de \textbf{\$0,0049} (aproximadamente 0,005 
unidades monetarias). Comparando este resultado con el \textit{threshold} de \textbf{0,01}, se evidencia que el sistema opera con un margen de eficiencia del 
\textbf{50\%} respecto al límite máximo permitido.


\newpage
\section{\hyperref[req:RNF-MANT-008]{RNF-MANT-008} - Mantenibilidad}
\label{appendix:posthog-metrics}

Para validar las hipótesis de valor planteadas en la sección \ref{sec:valor-hipotesis} y asegurar la mantenibilidad del sistema a través de la 
toma de decisiones basada en datos, se ha instrumentado la aplicación mediante la plataforma de analítica de producto \textit{PostHog}.

\subsubsection{Eventos de Captura}

Esta instrumentación se ha diseñado para medir el éxito y la fricción en los flujos críticos de la aplicación. Se han definido los siguientes eventos específicos que permiten monitorizar el flujo del usuario 
desde que inicia una acción hasta que esta se completa con éxito:

\vspace{1cm}
\begin{table}[H]
\scriptsize
\begin{tabularx}{\textwidth}{|l|X|}
\hline
\rowcolor[HTML]{EFEFEF} 
{\footnotesize \textit{Nombre de evento}} & {\footnotesize \textit{Descripción}} \\ \hline
{\scriptsize \texttt{START\_MANUAL\_NEW\_CONTACT}} & Inicio de creación manual de un contacto. \\ \hline
{\scriptsize \texttt{CANCEL\_MANUAL\_NEW\_CONTACT}} & Cancelación del flujo de creación manual de un contacto. \\ \hline
{\scriptsize \texttt{MANUAL\_NEW\_CONTACT\_SUCCESS}} & Contacto creado correctamente de forma manual. \\ \hline
{\scriptsize \texttt{START\_RECORDING\_NEW\_CONTACT}} & Inicio de grabación por voz para nuevo contacto. \\ \hline
{\scriptsize \texttt{EDIT\_TEXT\_RECORDING\_NEW\_CONTACT}} & Edición del texto transcrito del dictado. \\ \hline
{\scriptsize \texttt{CANCEL\_RECORDING\_NEW\_CONTACT}} & Cancelación de la creación por dictado de creación de un contacto. \\ \hline
{\scriptsize \texttt{RECORDING\_NEW\_CONTACT\_SUCCESS}} & Contacto creado correctamente por dictado. \\ \hline
{\scriptsize \texttt{START\_IMPORTING\_NEW\_CONTACT}} & Inicio de importación de un contacto. \\ \hline
{\scriptsize \texttt{IMPORTING\_NEW\_CONTACT\_SUCCESS}} & Contacto importado correctamente. \\ \hline
{\scriptsize \texttt{START\_MANUAL\_EDIT\_CONTACT}} & Inicio de edición manual de un contacto. \\ \hline
{\scriptsize \texttt{CANCEL\_MANUAL\_EDIT\_CONTACT}} & Cancelación de la edición del contacto. \\ \hline
{\scriptsize \texttt{MANUAL\_EDIT\_CONTACT\_SUCCESS}} & Contacto editado correctamente. \\ \hline
{\scriptsize \texttt{DELETE\_MANUAL\_CONTACT}} & Eliminación de un contacto. \\ \hline
{\scriptsize \texttt{VIEW\_CONTACT}} & Visualización de la ficha de un contacto. \\ \hline
{\scriptsize \texttt{START\_INTERACTION\_RECORDING}} & Inicio de grabación de una interacción (voz). \\ \hline
{\scriptsize \texttt{CANCEL\_INTERACTION\_RECORDING}} & Cancelación de la grabación de interacción. \\ \hline
{\scriptsize \texttt{EDIT\_TEXT\_RECORDING\_INTERACTION}} & Edición del texto de la interacción grabada. \\ \hline
{\scriptsize \texttt{INTERACTION\_RECORDING\_SUCCESS}} & Interacción grabada correctamente. \\ \hline
{\scriptsize \texttt{ASSIGN\_UNASSIGNED\_INTERACTION}} & Asignación de una interacción sin asignar. \\ \hline
{\scriptsize \texttt{SELECT\_CONTACT\_TO\_ASSIGN\_INTERACTION}} & Selección de contacto para asignar la interacción. \\ \hline
{\scriptsize \texttt{ASSIGN\_INTERACTION\_SUCCESS}} & Interacción asignada correctamente. \\ \hline
{\scriptsize \texttt{VIEW\_INTERACTION}} & Visualización del detalle de una interacción. \\ \hline
{\scriptsize \texttt{DELETE\_INTERACTION}} & Eliminación de una interacción. \\ \hline
{\scriptsize \texttt{MANUAL\_PROCESS\_INTERACTION}} & Procesamiento manual de una interacción. \\ \hline
\end{tabularx}
\caption{Eventos de PostHog capturados en la aplicación}
\label{tab:eventos-posthog}
\end{table}

\subsubsection{Anonimización de Datos}

En cumplimiento con los principios de \textit{Privacidad desde el Diseño}, se ha implementado una capa de abstracción en el envío de métricas 
para que se desabiliten los campos de información sensible o geolocalización del usuario. Como se observa en el siguiente fragmento de código, se han desactivado explícitamente los 
campos de \textit{GeoIP} (ciudad, país, latitud/longitud), la dirección IP, la zona horaria y la configuración de idioma local (\textit{locale}). Esta medida asegura que las 
métricas obtenidas sean puramente operacionales y anónimas.

{\scriptsize
\begin{lstlisting}[language=TypeScript, caption=Implementación de abstracción de eventos con anonimización de metadatos.]
posthog.capture(event_name, {
  $geoip_disable: true,
  $ip: "",
  $locale: false,
  $timezone: false,
  // Desactivación de todos los campos de geolocalización
  $geoip_city_name: false,
  $geoip_country_code: false,
  ...properties,
});
\end{lstlisting}}

\subsubsection{Verificación del Flujo de Datos}

La Figura \ref{fig:evidencia-posthog-stream} muestra el flujo de eventos en tiempo real capturado durante una prueba de uso. Se confirma la recepción de los eventos 
lo que permite validar que el sistema de monitorización es plenamente funcional y capaz de impulsar el ciclo de aprendizaje validado de la metodología \textit{Lean Startup} (fases \textit{Medir-Aprender}).

\vspace{.5cm}
\begin{figure}[H]
    \centering
    \includegraphics[width=.8\textwidth]{figures/posthog-event-stream.png}
    \caption{Flujo de eventos recibido en el panel de PostHog.}
    \label{fig:evidencia-posthog-stream}
\end{figure}