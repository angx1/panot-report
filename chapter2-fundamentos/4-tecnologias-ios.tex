%%%%%%%%%%%%%%%%%%%%%%%%%%%%%%%%%%%
% TECNOLOGÍAS PARA EL DESARROLLO DE APLICACIONES iOS

\section{Tecnologías para el Desarrollo de Aplicaciones para iOS}

\subsection{Desarrollo Nativo}

El desarrollo nativo de aplicaciones para iOS se basa en un ecosistema integrado de lenguajes, 
herramientas y frameworks diseñados y mantenidos por Apple para garantizar la calidad, seguridad y consistencia de las aplicaciones 
en sus dispositivos.

Dos lenguajes de programación principales conforman el núcleo del desarrollo nativo: \textit{Swift} y 
\textit{Objective-C}. \textit{Swift}, introducido en 2014, es el lenguaje moderno y recomendado para 
nuevos proyectos, ofreciendo tipado estático, gestión automática de memoria mediante referencia contadora 
(\textit{ARC}) y una sintaxis expresiva que mejora la seguridad y la productividad del desarrollador. 
\textit{Objective-C} es el lenguaje tradicional usado por Apple, sigue siendo compatible y se utiliza principalmente en 
proyectos legados o para integración con frameworks de bajo nivel.

El entorno de desarrollo integrado por excelencia es \textit{Xcode}, el IDE oficial y exclusivo para iOS. 
Xcode proporciona todas las herramientas necesarias para el flujo completo de desarrollo: editor de código, 
depurador, simulador de dispositivos y utilidades como \textit{Interface Builder} para el diseño visual 
de interfaces, \textit{Instruments} para análisis de rendimiento, y gestión de certificados y perfiles 
de aprovisionamiento imprescindibles en el proceso de firma y distribución de aplicaciones.

En cuanto a la construcción de la interfaz de usuario, existen dos frameworks principales: 
\textit{SwiftUI} y \textit{UIKit}. \textit{SwiftUI}, presentado en 2019, representa el enfoque 
declarativo y moderno donde la interfaz se define desde código Swift utilizando un paradigma 
funcional y reactivo. Por su parte, \textit{UIKit} es el framework tradicional fundamentado en un 
enfoque imperativo y el patrón Modelo-Vista-Controlador (MVC), manteniendo hoy en día una presencia 
relevante tanto en desarrollos existentes como en escenarios que requieren un control más 
directo del sistema.

Apple incorpora también diversas herramientas adicionales usadas frecuentemente en el desarrollo de aplicaciones 
para iOS. Por un lado, \textit{CocoaPods} y \textit{Swift Package Manager} (SPM) que permiten la gestión de dependencias 
externas. Por otro lado, \textit{Core Data} y \textit{CloudKit} que facilitan la persistencia de datos local y 
sincronización en la nube respectivamente. O, por otro lado, \textit{Combine}, que es un framework introducido 
junto con SwiftUI que permite la gestión de flujos de datos asíncronos.

La combinación de estos lenguajes, herramientas y frameworks permite la creación de aplicaciones 
iOS de alto rendimiento y experiencia de usuario optimizada, garantizando compatibilidad, seguridad 
y aprovechamiento completo de las capacidades nativas del sistema operativo y el hardware de Apple.

\subsection{Frameworks Multiplataforma}

Además del desarrollo nativo, existen varios frameworks multiplataforma que permiten desarrollar aplicaciones 
para iOS junto con otras plataformas (principalmente Android) desde una base de código compartida.
{\small
\begin{itemize}
    \item \textit{React Native}, desarrollado por Meta, permite crear aplicaciones móviles utilizando JavaScript y React. 
    El framework utiliza un puente nativo que comunica el código JavaScript con componentes nativos de cada plataforma, 
    permitiendo acceso a APIs nativas mientras se comparte la mayor parte de la lógica de negocio.
    
    \item \textit{Expo}, construido sobre React Native, proporciona un conjunto de herramientas y servicios que simplifican 
    el desarrollo, incluyendo un runtime unificado, APIs listas para usar y un sistema de compilación en la nube. Expo 
    reduce significativamente la complejidad de configuración del proyecto y facilita el despliegue, aunque con algunas 
    limitaciones en el acceso a funcionalidades nativas avanzadas.
    
    \item \textit{Flutter}, desarrollado por Google, utiliza el lenguaje \textit{Dart} y un motor de renderizado propio que 
    compila a código nativo. Flutter construye la interfaz de usuario desde cero en cada plataforma, evitando la 
    necesidad de componentes nativos del sistema operativo y proporcionando mayor consistencia visual entre plataformas.
\end{itemize}
}
Otras alternativas multiplataforma incluyen \textit{Xamarin} (ahora \textit{.NET MAUI}) que utiliza C\# y .NET, 
\textit{Ionic} que combina tecnologías web (HTML, CSS, JavaScript) con capacidades nativas mediante \textit{Cordova}, 
y \textit{Unity} para aplicaciones que requieren capacidades gráficas avanzadas, esencialmente, videojuegos.

La elección entre desarrollo nativo y multiplataforma depende de factores como requisitos de rendimiento, necesidad 
de acceso a funcionalidades nativas avanzadas, tiempo de desarrollo, mantenimiento a largo plazo o recursos del equipo.

\subsection{Proceso de Desarrollo y Distribución}

El ciclo de vida de una aplicación iOS desde el desarrollo hasta su distribución sigue un proceso estructurado:

\begin{enumerate}
    \item \textit{Desarrollo}: Durante esta fase, el código se compila y ejecuta en simuladores iOS o dispositivos físicos mediante 
    perfiles de desarrollo.
    
    \item \textit{Pruebas internas}: Para pruebas internas, las aplicaciones, una vez compiladas, es posible su distribución mediante 
    \textit{TestFlight}, plataforma que permite a desarrolladores invitar hasta 10.000 beta testers externos sin 
    necesidad de certificados adicionales.
    
    \item \textit{Archive}: El proceso de \textit{Archive} genera un paquete de distribución de la aplicación (\textit{.ipa}) optimizado 
    y firmado con los certificados de distribución \footnote{Los certificados de distribución son credenciales digitales emitidas por 
    Apple que permiten identificar y autenticar al desarrollador, garantizando que la aplicación proviene de una fuente confiable y 
    verificada. Son esenciales para firmar digitalmente las aplicaciones antes de su distribución en la App Store.}.
    Esta versión archivada puede subirse a \textit{App Store Connect}, portal web de Apple para gestión de aplicaciones, donde se 
    configura información de la aplicación, capturas de pantalla, descripciones y metadatos requeridos para la publicación.
    
    \item \textit{App Review}: Una vez hecha la petición de subida, este proceso de revisión de Apple verifica que el producto cumple con 
    las directrices de la App Store, incluyendo seguridad, privacidad, calidad técnica y contenido. Una vez aprobada, la aplicación 
    está disponible para distribución pública o privada según la configuración establecida.
    
    \item \textit{Distribución}: La distribución puede realizarse mediante tres canales principales:
    {\small
    \begin{itemize}
        \item \textbf{App Store}: Para usuarios finales a través de la tienda oficial de Apple.
        \item \textbf{Distribución empresarial (\textit{Enterprise})}: Permite a las organizaciones internas distribuir aplicaciones privadas a sus empleados, fuera de la App Store.
        \item \textbf{Distribución ad-hoc}: Permite instalar la aplicación en un número limitado de dispositivos específicos, identificados mediante perfiles de aprovisionamiento
        \footnote{Los perfiles de aprovisionamiento son archivos que vinculan a un desarrollador y su aplicación con una cuenta de desarrollador, dispositivos y servicios autorizados}.
    \end{itemize}
    }
\end{enumerate}
