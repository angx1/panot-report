
\section{Gestión de la Inteligencia Relacional}

Los \textit{Sistemas de Gestión de Relaciones con Clientes} (\acrshort{crm}) convencionales, si bien útiles en entornos corporativos para la gestión masiva de clientes, presentan 
limitaciones significativas cuando se trata de capturar la complejidad inherente a las relaciones humanas. Estos sistemas 
operan principalmente con información descontextualizada, almacenando datos de contacto de manera estática y tabular sin considerar su evolución temporal ni su contexto situacional.

\subsection{Aplicaciones Similares}

El paradigma actual de la gestión de contactos ha evolucionado desde simples agentas digitales, a lo que ahora se denominan sistemas de \textit{Gestión de Relaciones Personales} (\acrshort{prm}).
Sin embargo, al analizar las soluciones líderes en el mercado, se han identificado patrones de diseño que limitan la capacidad de capturar la esencia de las relaciones humanas.

Aplicaciones de código abierto como \textit{Monica} o soluciones más comerciales como \textit{Dex} o \textit{Clay} operan bajo la misma premisa de transformar un contacto en una entidad más compleja 
y dinámica, no obstante, transfieren toda la carga cognitiva y administrativa al usuario, lo que inyecta cierta fricción a la hora de capturar información. 

Algunas de estas herramientas introducen el uso de modelos de lenguaje para enriquecer automáticamente los perfiles, extrayendo datos de redes sociales y correos electrónicos. 
Esto supone una propuesta de valor muy fuerte, pero, aunque reduzcan la fricción de la insercción de datos, su enfoque es fundamentalmente 'archivístico'. Al operar sobre este tipo de fuentes de datos, carecen de 
un contexto situacional profundo, es decir, saben donde trabaja un contacto, pero no saben cómo se sienten respecto a su nuevo empleo o qué matíz emocional tuvo la última conversación.

El vacío que se considera que estas aplicaciones no logran cubrir radica en la disonancia entre la naturaleza de las relaciones humanas y la arquitectura 
de las bases de datos que las contienen. Es decir, PANOT busca funcionar como una extensión cognitiva, y no una agenda perfecta. Por lo que la hipótesis de valor 
que enmarca PANOT, se posicionaría en el registro de \textit{Inteligencia Relacional}.

\subsection{Inteligencia Relacional en el Contexto de la Inteligencia Artificial}

La Inteligencia Relacional presenta un nuevo paradigma evolutivo en la gestión de información personal y profesional, introduciendo el
concepto de dinamismo contextual, donde cada relación evoluciona continuamente reflejando cambios en intereses, preferencias y circunstancias vitales. 
Este enfoque reconoce que las relaciones que tenemos no son entidades fijas, sino procesos complejos que cambian según un contexto temporal y situacional.

Para poder comprender el alcance de esta premisa, es necesario enmarcar el concepto dentro del ecosistema más amplio
de la Inteligencia Artificial y analizar cómo se diferencia con los paradigmas tradicionales. 

La diferencia fundamental entre la Inteligencia Relacional y los paradigmas tradicionales de Inteligencia Artificial (\acrshort{ia}) radica en que, mientras estos últimos 
operan principalmente mediante la aproximación estadística de distribuciones de datos y patrones (optimizando funciones de pérdida sobre grandes volúmenes 
de datos descontextualizados), la Inteligencia Relacional se fundamenta en la construcción y manipulación de representaciones 
estructurales de relaciones que permiten la generalización cruzada y la \gls{inferencia-analogica}. 
La Inteligencia Relacional captura la estructura relacional subyacente que puede transferirse entre dominios aparentemente no relacionados, 
tal como ocurre en el razonamiento humano, creando un contexto de conocimiento más amplio y generalizable o especfíco para un dominio en concreto.

La investigación en Inteligencia Relacional demuestra capacidades que van más allá del aprendizaje estadístico tradicional. 
En 2022, se publicó un estudio de la Universidad de Edimburgo\cite{doumas2022theory} en el que se muestra cómo un modelo computacional puede aprender representaciones relacionales 
estructuradas y realizar \gls{generalizacion-de-cero-disparos} entre dominios completamente diferentes, como la transferencia de 
conocimiento entre videojuegos. Esta capacidad de generalización permite que el sistema aprenda a reconocer y comprender relaciones 
entre entidades en contextos completamente diferentes.

Traspasando la analogía de los videojuegos al contexto de PANOT, la Inteligencia Relacional como se menciona en \cite{doumas2022theory} permite que el sistema
aprenda a reconocer y comprender patrones relacionales estructurados entre personas, eventos y contextos, más allá de las asociaciones estadísticas superficiales. 
En lugar de simplemente almacenar datos estáticos de contactos, el sistema de PANOT podría construir representaciones relacionales dinámicas que capturan la estructura subyacente 
de las relaciones del usuario que use la plataforma.

Para ilustrar este proceso, consideremos un ejemplo práctico del flujo de procesamiento relacional que realizaría el sistema de PANOT:

\vspace{1cm}
\noindent \textbf{\textit{Entrada:}} El usuario captura la siguiente interacción mediante una nota de voz: ``Acabo de almorzar con María. Está muy emocionada 
porque ha conseguido un nuevo trabajo como diseñadora en una startup tecnológica. Le interesa especialmente el trabajo remoto 
y mencionó que está buscando un piso más cerca de su nueva oficina. Hablamos de proyectos de diseño colaborativo y se mostró 
muy receptiva a la idea de trabajar juntos en el futuro.''

\vspace{.5cm}
\noindent \textbf{\textit{Procesamiento:}} PANOT procesa esta entrada extrayendo múltiples capas de información relacional 
estructurada:

{\footnotesize
\begin{itemize}
    \setlength{\itemsep}{0pt}
    \setlength{\parsep}{0pt}
    \setlength{\topsep}{0pt}
    \setlength{\partopsep}{0pt}
    \item \textit{Evento}: almuerzo social de contexto informal
    \item \textit{Cambio de estado}: transición profesional — nuevo trabajo como diseñadora
    \item \textit{Cambio de preferencias}: prioridad hacia trabajo remoto
    \item \textit{Necesidad emergente}: búsqueda de vivienda
    \item \textit{Relaciones}: interés común en proyectos de diseño colaborativo, receptividad a colaboración en el futuro 
    \item \textit{Contexto temporal}: estado emocional positivo, momento de transición vital
\end{itemize}}


\noindent El sistema construye una representación relacional estructurada que conecta estas entidades (usuario-contacto)
mediante relaciones semánticas. A modo ilustrativo, esta sería una posible representación en formato {\footnotesize \texttt{JSON}}:

\clearpage
{\scriptsize
\begin{lstlisting}[language=JSON, caption=Posible representación relacional estructurada del contacto «María».]
{
  "personal": {
    "necesidad_actual": "búsqueda de vivienda",
    "preferencia_ubicación": "cerca de la nueva oficina",
    "estado_emocional": "emocionada"
  },
  "profesional": {
    "rol": "diseñadora",
    "empresa": "startup tecnológica",
    "intereses": ["trabajo remoto"],
    "estado": "transición laboral reciente"
  },
  "relación": {
    "última_interacción": "almuerzo informal",
    "intereses_comunes": ["diseño colaborativo", "proyectos futuros"],
    "dinámica": "receptiva a colaboración"
  }
}
\end{lstlisting}}

\vspace{.5cm}
\noindent \textbf{\textit{Salida:}} PANOT actualiza dinámicamente el contexto de «María», integrando la nueva información estructurada directamente en su perfil relacional:

{\small
\begin{itemize}
    \setlength{\itemsep}{0pt}
    \setlength{\parsep}{0pt}
    \setlength{\topsep}{0pt}
    \setlength{\partopsep}{0pt}
    \item \textit{Contexto Personal}: Se actualiza la información sobre su situación personal con ``búsqueda de vivienda'' y su preferencia de ubicación, además de capturar el estado emocional positivo asociado al cambio de trabajo.
    \item \textit{Contexto Profesional}: Se modifica la información sobre su situación profesional para incluir el nuevo rol de ``diseñadora en startup tecnológica'' y se añade el ``trabajo remoto'' como un interés profesional clave en esta etapa.
    \item \textit{Contexto de la Relación}: Se modifica la información sobre la dinámica de la relación registrando los ``proyectos de diseño colaborativo'' como un nuevo eje de interés común y actualizando el estado de la relación hacia una posible colaboración futura.
\end{itemize}}

\noindent Así, el contacto de María dentro de la base de datos de PANOT quedaría como un conjunto de nodos interconectados que representan 
eventos, gustos, situaciones, necesidades, etc. abstrayéndo el complejo contexto de la relación en una representación más simplificada \ref{fig:fig-gf-maria}.

\begin{figure}[H]
  \centering
  \includegraphics[width=1\textwidth]{figures/grafo-contexto-maria.png}
  \caption{\label{fig:fig-gf-maria} Representación orientativa usando MemGraph Lab del contexto del contacto «María» en formato de grafo relacional.}
\end{figure}


\subsection{Grafos Contextuales en Sistemas de Agentes}
\label{sec:grafos-contextuales}

La representación relacional estructurada que utiliza PANOT encuentra un paralelismo arquitectónico significativo con sistemas de agentes 
de inteligencia artificial que emplean grafos contextuales como mecanismo de memoria persistente. Estos sistemas, inspirados en arquitecturas 
como GraphRAG (Graph Retrieval-Augmented Generation), implementan grafos de conocimiento que permiten a los agentes acceder de manera 
eficiente a información estructurada y realizar razonamientos complejos mediante la navegación de relaciones semánticas.

La eficiencia superior de las estructuras de grafo para la memoria de los agentes radica en la diferencia fundamental entre el modelo de 
acceso a datos relacionales y el modelo de navegación por grafos. En las bases de datos relacionales, recuperar información sobre 
relaciones entre entidades requiere realizar múltiples operaciones que pueden cruzar tablas diferentes, lo cual implica escanear 
índices y realizar comparaciones entre grandes volúmenes de datos. La complejidad de estas operaciones crece exponencialmente con el número 
de relaciones involucradas, resultando en tiempos de consulta que pueden ser $\mathcal{O}(n \log n)$ o peor cuando se requieren múltiples uniones de tablas 
anidadas. 

Por el contrario, en una estructura de grafo, acceder a los vecinos directos de un nodo, es decir, recuperar todas las relaciones 
de una entidad, es una operación de complejidad $\mathcal{O}(1)$ en promedio, ya que las conexiones están almacenadas directamente como parte de la 
estructura del nodo mediante listas de adyacencia o estructuras similares. Esta diferencia es crítica para sistemas de agentes que requieren 
acceso frecuente y rápido a información relacional.

Como ejemplo, si queremos que nuestro sistema haga una consulta relacional para recuperar ``todos los eventos relacionados con 
María y sus intereses comunes'' podría requerir de múltiples uniones de las tablas de contactos, eventos, intereses y relaciones en el caso
en el que se trate de un sistema de relacional. Por el contrario, en un grafo contextual, esta misma información se obtiene mediante una simple navegación 
a través de las aristas conectadas al nodo de María, accediendo directamente a los nodos adyacentes sin necesidad de realizar búsquedas complejas.

En el contexto de PANOT, el grafo relacional que representa cada contacto y sus interacciones funciona análogamente al grafo contextual de 
un sistema agéntico: ambos proporcionan una estructura que permite acceso eficiente a información relevante, razonamiento sobre relaciones 
complejas y actualización dinámica del conocimiento. Esta arquitectura permite que PANOT no solo almacene información sobre contactos, sino 
que también pueda realizar inferencias relacionales, generalizar patrones entre relaciones y adaptarse continuamente a la evolución de las 
interacciones humanas.

