\section{Principios de Privacidad y Eficiencia por Diseño}

El desarrollo de sistemas de software modernos, especialmente aquellos que procesan información personal y 
utilizan modelos de lenguaje, requieren de la integración de principios 
fundamentales desde las primeras etapas del diseño. En este apartado se presentan dos marcos conceptuales 
esenciales: la \textit{Privacidad desde el Diseño}, enmarcada en el contexto normativo español, y 
la \textit{Eficiencia por Diseño}, fundamentada en las mejores prácticas de arquitectura de software modernas.

\subsection{Principio de Privacidad}
\label{sec:principio-privacidad}

La \textit{Privacidad desde el Diseño} (Privacy by Design, PbD) constituye un enfoque proactivo que integra 
la protección de datos personales desde las primeras fases de desarrollo de productos, servicios o procesos. 
Este concepto, desarrollado inicialmente por Ann Cavoukian en la década de los 90 y reconocido internacionalmente 
en la 32ª Conferencia Internacional de Comisionados de Protección de Datos y Privacidad celebrada en Jerusalén en 2010, 
ha sido consolidado legalmente en el Reglamento General de Protección de Datos (RGPD) mediante su artículo 25, 
que establece la obligación de implementar medidas técnicas y organizativas apropiadas para garantizar la 
protección de datos desde el diseño y por defecto en productos y servicios de software.

La Agencia Española de Protección de Datos (AEPD), en su \textit{Guía de Privacidad desde el Diseño} 
\cite{aepd2019privacidad}, identifica siete principios fundacionales que deben guiar el desarrollo de 
sistemas que procesan datos personales:

{\small
\begin{enumerate}
    \item \textit{Proactivo, no reactivo; preventivo, no correctivo}: La privacidad debe anticiparse a los 
    riesgos antes de que se materialicen, implementando medidas preventivas en lugar de soluciones correctivas 
    posteriores.
    
    \item \textit{La privacidad como configuración predeterminada}: Los sistemas deben proteger 
    los datos personales por defecto, sin requerir acción adicional del usuario. La configuración más 
    privada debe ser la opción predeterminada.
    
    \item \textit{Privacidad incorporada en la fase de diseño}: Las medidas de protección deben integrarse 
    desde el inicio del desarrollo, evitando soluciones añadidas posteriormente que puedan ser menos efectivas 
    o generar fricciones en la experiencia del usuario.
    
    \item \textit{Funcionalidad total}: Los objetivos de privacidad no deben comprometer 
    la funcionalidad del sistema. El diseño debe equilibrar ambos aspectos, garantizando que la protección de 
    datos y las funcionalidades esenciales coexistan sin sacrificios mutuos.
    
    \item \textit{Aseguramiento de la privacidad en todo el ciclo de vida}: La protección de datos debe 
    mantenerse durante todas las etapas del ciclo de vida del sistema, desde su concepción hasta su retirada, 
    incluyendo desarrollo, puesta en producción, despliegue, mantenimiento y eliminación.
    
    \item \textit{Visibilidad y transparencia}: Los usuarios deben tener información clara y accesible sobre las 
    prácticas de tratamiento de datos, fomentando la confianza mediante la transparencia en los procesos.
    
    \item \textit{Mantener un enfoque centrado en el usuario}: El diseño 
    debe respetar las preferencias y necesidades de privacidad de los usuarios, otorgándoles control efectivo 
    sobre sus datos personales y manteniendo un enfoque que priorice sus derechos y expectativas.
\end{enumerate}
}

La implementación de estos principios requiere un enfoque de \textit{Privacy Engineering} o ingeniería de 
la privacidad, que traduce los principios conceptuales en medidas técnicas concretas durante las diversas fases de 
desarrollo. La guía de la AEPD identifica estrategias de diseño específicas, entre las que 
destacan las siguientes estrategias de diseño:

{\small
\begin{enumerate}
    \item \textbf{\textit{Minimizar}}: Recopilar únicamente los datos estrictamente necesarios.
    \item \textbf{\textit{Ocultar}}: Proteger la información sensible empleando técnicas de ofuscación o desvinculación.
    \item \textbf{\textit{Separar}}: Gestionar los datos de forma que se eviten correlaciones indebidas o perfilados del usuario.
    \item \textbf{\textit{Abstraer}}: Limitar el nivel de detalle de los datos personales tratados.
    \item \textbf{\textit{Informar}}: Comunicar de manera clara a los usuarios sobre el tratamiento de sus datos.
    \item \textbf{\textit{Controlar}}: Garantizar un adecuado control de acceso y uso de la información, así como permitir a los usuarios gestionar la recogida, tratamiento, usos y comunicaciones de sus datos personales.
    \item \textbf{\textit{Cumplir}}: Asegurar que los tratamientos de datos personales se llevan a cabo conforme a los procedimientos y requisitos legales aplicables.
    \item \textbf{\textit{Demostrar}}: Según el artículo 24 de la LGPD, poder evidenciar el cumplimiento normativo a través de documentación y auditorías, tanto frente a los usuarios como ante las autoridades de supervisión.
\end{enumerate}
}

\subsection{Principio de Eficiencia}

La \textit{Eficiencia por Diseño} (Efficiency by Design) constituye un enfoque arquitectónico que integra 
la optimización de recursos computacionales desde las primeras etapas del diseño, garantizando que los sistemas 
puedan gestionar su carga de trabajo utilizando la menor cantidad de recursos posibles sin comprometer la funcionalidad 
ni la experiencia de usuario. A diferencia de la optimización reactiva, que se aplica una vez que el sistema ya está 
en producción, la eficiencia por diseño requiere considerar aspectos de rendimiento, escalabilidad y consumo de recursos 
desde las fases de análisis y diseño arquitectónico.

En su libro \textit{Clean Architecture}, Robert C. Martin \cite{martin2017clean} aborda esta cuestión desde una perspectiva más fundacional, 
enfatizando la importancia de promover la eficiencia a largo plazo mediante el diseño de sistemas flexibles y mantenibles, a través de la 
separación de la lógica de negocio y los detalles técnicos. Esto se consigue con un diseño de capas donde las dependencias siempre 
apuntan hacia el interior, de modo que los cambios en aspectos técnicos, es decir, en capas más centrales, no impacten en las capas superiores de 
principios y reglas del sistema. De esta manera, se favorece la creación de sistemas eficientes, sostenibles y modulares, capaces de mantener su 
funcionalidad y adaptabilidad en el tiempo.

En la práctica, las empresas de software modernas implementan el principio de eficiencia por diseño mediante diversas 
estrategias arquitectónicas. Por ejemplo, la adopción de arquitecturas serverless y funciones como servicio (\acrshort{faas}) permite 
a empresas como Netflix optimizar el consumo de recursos computacionales, pagando únicamente por el tiempo de ejecución 
real en lugar de mantener servidores activos de forma continua. Otras empresas, como Spotify, han implementado 
arquitecturas de microservicios con auto-escalado horizontal, permitiendo que la infraestructura se adapte dinámicamente 
a la carga de trabajo sin sobre-provisionamiento de recursos. Asimismo, la implementación de estrategias de caching 
distribuido y \acrshort{cdn} (Content Delivery Network) en empresas como Amazon Web Services permite reducir la latencia y el 
consumo de ancho de banda mediante el almacenamiento de contenido en ubicaciones geográficamente cercanas a los usuarios finales.

\subsection{Aplicación en Sistemas con Modelos de Lenguaje}

La aplicación de los principios de privacidad y eficiencia por diseño adquiere particularidades específicas cuando 
se trata de sistemas que integran modelos de lenguaje, debido a la naturaleza sensible de los datos procesados 
y a los elevados costes computacionales asociados con el procesamiento de lenguaje natural.

En el contexto de aplicaciones que utilizan modelos de lenguaje, la protección de la privacidad presenta desafíos 
únicos derivados de la necesidad de procesar información personal, a menudo sensible, mediante servicios externos 
o infraestructuras en la nube. Para abordar estos desafíos, se han desarrollado diversas 
\textit{Tecnologías de Preservación de la Privacidad} (\acrshort{tpp}) que permiten procesar datos manteniendo su confidencialidad.

Una de las técnicas más prometedoras en este ámbito es el \textit{Cifrado Totalmente Homomórfico} 
(\acrshort{fhe}), un paradigma criptográfico que, diferencia del 
cifrado tradicional, donde los datos deben descifrarse antes de ser procesados, 
permite que un servidor o servicio externo procese información cifrada y devuelva resultados también 
cifrados, manteniendo la confidencialidad de los datos en todo momento. Este enfoque es especialmente 
relevante en sistemas que utilizan modelos de lenguaje en entornos como el sanitario \cite{tfg-mgsicilia}, 
ya que permitiría enviar datos sensibles sobre pacientes a servicios de procesamiento de lenguaje natural 
sin que el proveedor del servicio pueda acceder al contenido real de la información.

Sin embargo, el \acrshort{fhe} presenta limitaciones prácticas significativas en términos 
de rendimiento y eficiencia computacional ya que este tipo de operaciones requieren de modelos pre-entrenados con datos homomórficamente cifrados. 
Por esta razón, su aplicación en sistemas desplegados sigue siendo un área de investigación activa, 
aunque existen implementaciones experimentales que demuestran su viabilidad para casos de uso específicos.

Otras \acrshort{tpp} en sistemas con modelos de lenguaje incluyen la \textit{Minimización de datos}, enviando únicamente la información estrictamente necesaria 
para el procesamiento, la \textit{Ofuscación} de información sensible mediante técnicas de anonimización o 
pseudonimización, y el \textit{Procesamiento Local} cuando es posible, utilizando modelos de lenguaje más 
pequeños ejecutados directamente en el dispositivo del usuario para evitar la transmisión de datos a servicios externos.

En el caso de la eficiencia en este tipo de sistemas, una estrategia arquitectónica fundamental es la adopción de 
\textit{arquitecturas serverless} y \acrshort{faas}, que permiten ejecutar procesamiento 
de modelos de lenguaje de manera escalable y bajo demanda. Los servicios serverless eliminan la necesidad de 
mantener servidores activos de forma continua, permitiendo que la infraestructura se active únicamente cuando 
se requiere procesamiento, reduciendo significativamente los costes operativos. Plataformas como AWS Lambda, 
Google Cloud Functions o Azure Functions permiten ejecutar funciones que invocan modelos de lenguaje, pagando 
únicamente por el tiempo de ejecución real y los recursos computacionales consumidos durante cada invocación.

Además de la arquitectura serverless, la \textit{elección del modelo de lenguaje adecuado} constituye un factor 
crítico para optimizar la eficiencia en sistemas que procesan lenguaje natural. La selección del modelo debe 
equilibrar tres dimensiones fundamentales: la \textit{eficiencia computacional}, el \textit{coste económico} 
y el \textit{tiempo de respuesta}.

\subsection{Aplicación en PANOT}
\label{sec:aplicacion-privacidad-eficiencia-panot}

El caso del sistema que se ha desarrollado en este proyecto, se ha planteado un enfoque que respeta y aplica los 
principios de privacidad y eficiencia desde el diseño del sistema.

El principio de privacidad en un sistema como PANOT,
adquiere una relevancia especial ya que se manejan datos sobre relaciones interpersonales, preferencias y 
contextos situacionales de los contactos del usuario. Sin embargo, la aplicación de la privacidad desde 
el diseño en PANOT presenta particularidades específicas derivadas de su naturaleza como herramienta personal de 
gestión de contactos.

PANOT está diseñado como una aplicación personal donde el usuario gestiona y evoluciona el perfil de sus propios contactos. 
En este contexto, la privacidad require de proteger la información del usuario  
frente a accesos no autorizados y cumplir con las restricciones legales sobre datos especialmente sensibles dentro de lo que es posible. 

A nivel de infraestructura, como se explicará en el siguiente capítulo, se ha utilizado \textit{Supabase} como plataforma de backend. 
\textit{Supabase} tiene la particularidad de autogestionar estas medidas de seguridad a través de dos mecanismos: 

{\small
\begin{itemize}
    \item Por un lado, todos los datos se encriptan en reposo mediante \gls{aes-256} y en tránsito mediante el protocolo \gls{transport-layer-security}, 
garantizando la protección de la información en todas las etapas del ciclo de vida. 
    \item Y por otro lado, \textit{Supabase} proporciona \gls{row-level-security} y \gls{column-level-security}, permitiendo un control granular del acceso a los datos a nivel de 
fila y columna, lo que asegura que cada usuario solo pueda acceder a sus propios datos y a nada más. 
\end{itemize}
}

\textit{Supabase} además cumple con estándares de seguridad como \gls{soc-2-type-2}, proporcionando una base de cumplimiento normativo adicional. 

La aplicación de los principios de eficiencia por diseño en PANOT se ha materializado mediante la adopción de \textit{arquitecturas serverless} y de una 
metodología de desarrollo inspirada en el enfoque \textit{Lean Startup} propuesto por Eric Ries \cite{ries2011lean}. 
Este marco metodológico, fundamentado en el ciclo iterativo \gls{construir-medir-aprender}, establece que el 
desarrollo de productos debe priorizar la entrega de valor con el menor coste posible, en el menor tiempo posible, 
manteniendo la máxima calidad en las funcionalidades principales del sistema \ref{sec:filosofia-metodologia-desarrollo}.

En este contexto, durante el diseño arquitectónico de PANOT, se han orientando las decisiones técnicas hacia la construcción 
de una infraestructura escalable y eficiente desde sus fundamentos. Este enfoque ha permitido priorizar la funcionalidad 
esencial del sistema mientras se optimiza el consumo de recursos computacionales y económicos asociados con el alojamiento de la infraestructura y 
el uso de servicios externos.


