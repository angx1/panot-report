%%%%%%%%%%%%%%%%%%%%%%%%%%%%%%%%%%%
% FASES DE LA IMPLEMENTACIÓN

\clearpage
\section{Implementación del Sistema}

Como se mencionó en la sección \ref{sec:enfoque-lean}, el desarrollo de PANOT se centra en la producción de un artefacto que permita 
validar las hipótesis de valor. Es por esto que, a diferencia de un modelo incremental tradicional, la implementación de este sistema 
se ha llevado a cabo como un esfuerzo unificado para materializar el \acrshort{mvp}. Bajo esta premisa, la implementación de PANOT se ha 
organizado en ejes de desarrollo concurrentes para maximizar la velocidad de desarrollo y permitir el trabajo paralelo sobre las diferentes 
capacidades del sistema. 

Es esencial señalar que en este proyecto, el despliegue no es una fase posterior de la implementación, sino su culminación. La publicación 
de la aplicación \textit{"Semilla"} en una plataforma de distribución es el evento de cierre del proceso de construcción (Fase \textit{Construir}) y da comienzo 
a la fase de medición (Fase \textit{Medir}). Por ello, la implementación incluirá tareas como la instrumentación de métricas ya que, sin ellas, 
el \acrshort{mvp} no podría cumplir su propósito de validación de la hipótesis de valor.

\subsection{Configuración Inicial}




