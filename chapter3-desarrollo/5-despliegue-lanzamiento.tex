\clearpage
\section{Despliegue y Lanzamiento de PANOT}

Tras culminar con las fases de análisis, diseño e implementación, el proyecto entra en su fase clave de puesta en producción.
Este proceso no se limita únicamente a transferir el código a un servidor, sino que engloba una serie de verificaciones de seguridad, 
la preparación de artefactos nativos y la gestión administrativa ante las plataformas de distribución (en este caso la App Store de Apple).
El objetivo final es garantizar que el \acrshort{mvp} sea accesible para los usuarios finales bajo los estándares de calidad y privacidad definidos en los 
capítulos anteriores \ref{tab:non-functional-requirements} y los impuestos por las políticas de Apple.

Dada la naturaleza académica de este proyecto y el marco normativo de la Universidad Politécnica de Madrid, el despliegue está orientado a la 
materialización de un artefacto funcional ajustando ciertas capacidades comerciales para alinearse con los derechos de explotación de la institución.

\subsection{Preparación del Entorno y Adaptación Académica}

La transición hacia una versión productiva ha requerido una configuración específica de la infraestructura de backend en Supabase. A diferencia 
de una versión comercial abierta, y con el objetivo de garantizar la sostenibilidad del proyecto a futuro, se ha desplegado una versión 
que prioriza las capacidades de almacenamiento local y gestión de usuarios, contactos e interacciones descritas en la especificación del sistema. En este sentido, 
se ha optado por una configuración que prescinde de los módulos de pago recurrente y de las llamadas activas al sistema multi-agente en el entorno desplegado. 
Esta decisión técnica permite que tanto el evaluador y como futuros usuarios interactúen con una aplicación estable y fluida, centrada en la experiencia 
de usuario y la arquitectura local-first, sin riesgos operativos o económicos.

En cuanto al cliente móvil, el proceso de empaquetado se realizó de forma manual utilizando \textit{XCode}. Este flujo de trabajo se inicia con la 
generación del proyecto nativo de iOS mediante el comando de pre-construcción ({\footnotesize \texttt{npx expo prebuild --platform ios}}) de Expo, integrando todos 
los módulos que componen el proyecto. Una vez dentro de \textit{XCode}, se procedió a la firma del código mediante certificados de distribución oficiales, 
paso indispensable para generar el archivo con extensión .ipa (iOS App Store Package). Este archivo, constituye la entrega 
final y representaría el esfuerzo de ingeniería realizado listo para su distribución.

\subsection{Proceso de Publicación con App Store Connect}

\subsection{Estrategia de Lanzamiento y Visibilidad}
