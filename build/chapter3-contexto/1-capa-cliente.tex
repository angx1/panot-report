
\section{Capa del Cliente Móvil}

Para el desarrollo de la aplicación móvil PANOT se ha seleccionado \textit{Expo}, un framework construido sobre 
\textit{React Native} que simplifica significativamente el proceso de desarrollo de aplicaciones multiplataforma. 
Esta elección se fundamenta en varios factores clave:

{\small
\begin{itemize}
    \item \textbf{Rapidez de desarrollo}: \textit{Expo} proporciona un conjunto de herramientas e Interfaces de Programación de Aplicaciones (\acrshort{api}s) preconfiguradas que 
    reducen la complejidad de configuración del proyecto y aceleran el tiempo de desarrollo, permitiendo enfocar 
    los esfuerzos en la implementación de las funcionalidades. Además \textit{Expo} es un framework basado en 
    \textit{React}, lo que facilita el desarrollo modular, la reutilización de componentes y el tipado de los artefactos del proyecto debido a la naturaleza de este lenguaje.
    
    \item \textbf{Compatibilidad multiplataforma}: Aunque el proyecto se centra inicialmente en iOS, \textit{Expo} facilita 
    la extensión futura a Android con mínimos cambios en el código base, garantizando una base sólida para el crecimiento 
    del producto.
    
    \item \textbf{Ecosistema Open Source}: \textit{Expo} cuenta con una comunidad activa y documentación muy completa, lo que 
    facilita el aprendizaje y la resolución de problemas.
    
    \item \textbf{Acceso a funcionalidades nativas}: A través de módulos nativos y \acrshort{api}s expuestas por \textit{Expo}, se mantiene 
    acceso a capacidades del dispositivo como notificaciones push, o almacenamiento local, sin requerir la 
    complejidad del desarrollo nativo puro.
\end{itemize}
}

Se analizaron otras alternativas para el desarrollo del cliente móvil: en primer lugar, el desarrollo nativo con \textit{Swift} y \textit{SwiftUI}, que habría ofrecido un rendimiento 
mayor y acceso completo a las capacidades de iOS, pero se descartó por su mayor complejidad de configuración, tiempo de desarrollo más extenso y 
limitación a una única plataforma, lo que no se alineaba con los objetivos de eficiencia y escalabilidad del proyecto. Asimismo, se consideró \textit{Flutter}, un framework desarrollado por \textit{Google} 
que permite el desarrollo de aplicaciones multiplataforma con \textit{Dart} y compilación a código nativo, similar a \textit{Expo}, pero también se descartó por la poca familiaridad 
con este lenguaje.