
\section{Capa de Datos y Persistencia Local}
\label{sec:capa-datos}

Uno de los requisitos fundamentales de PANOT es garantizar la funcionalidad de la aplicación incluso en ausencia 
de conectividad a Internet, permitiendo a los usuarios capturar interacciones y gestionar sus contactos de manera continua 
independientemente de si tienen conexión o no. Para materializar este requisito, se ha adoptado un enfoque \textit{local-first}, 
donde los datos se almacenan localmente en primer lugar, sincronizándose con el servidor cuando la conectividad 
está disponible.

La implementación de este patrón se ha realizado mediante la combinación de \textit{Legend State}, una librería de gestión 
de estado reactiva y eficiente, junto con las capacidades nativas de almacenamiento local proporcionadas por Expo. \textit{Legend State} 
proporciona un sistema de observables que permite mantener un estado global de los datos sincronizado entre los componentes de la aplicación, 
mientras que \textit{Expo} ofrece \acrshort{api}s para el almacenamiento persistente en el dispositivo mediante \textit{AsyncStorage}.

El flujo de trabajo implementado sigue el siguiente patrón: cuando el usuario realiza una acción (por ejemplo, registrar 
una interacción), los datos se almacenan inmediatamente en el estado local gestionado por \textit{Legend State}. 
Esta operación es instantánea y no requiere conexión a internet. Posteriormente, en segundo plano, 
la aplicación intenta sincronizar estos datos con el servidor de \textit{Supabase} en cuanto se detecta conectividad disponible. Si la sincronización 
falla temporalmente, los datos permanecen en el dispositivo y se reintenta automáticamente cuando la conexión se restablece, 
garantizando que ninguna información se pierda.

Para la persistencia en la nube, se ha seleccionado, como ya se ha mencionado, \textit{Supabase} como infraestructura de backend, que proporciona una base de 
datos \textit{PostgreSQL} gestionada junto con \acrshort{api}s REST y en tiempo real. Esta elección se fundamenta en:

{\small
\begin{itemize}
    \item \textbf{Simplicidad de integración}: \textit{Supabase} proporciona un cliente JavaScript/TypeScript que se integra 
    naturalmente con React Native y Expo.
    
    \item \textbf{Escalabilidad}: \textit{PostgreSQL} es un motor de bases de datos relacional robusto y escalable, capaz de 
    gestionar grandes volúmenes de datos y relaciones complejas entre entidades.
    
    \item \textbf{Sincronización en tiempo real}: \textit{Supabase} ofrece capacidades de suscripción a cambios en tiempo real mediante \gls{websocket}s (\textit{Supabsae Realtime}), 
    permitiendo que actualizaciones realizadas localmente se reflejen automáticamente en la base de datos.
    
    \item \textbf{Seguridad integrada}: Como se mencionó en el capítulo anterior \ref{sec:aplicacion-privacidad-eficiencia-panot}, \textit{Supabase} proporciona mecanismos 
    de seguridad a nivel de fila y columna, y encriptación de datos en reposo y en tránsito, que garantizan el aislamiento de datos entre usuarios y encriptación de datos
    en todo su ciclo de vida.

    \item \textbf{Coste de uso}: el plan gratuito de \textit{Supabase} ofrece llamas ilimitadas a su \acrshort{api}, gestión de hasta 500.000 usuarios activos, 
    y hasta 500MB de almacenamiento de datos, lo que supone una base sólida para hacer posible el desarrollo del proyecto.  
\end{itemize}
}

No se han considerado alternativas a \textit{Supabase} para la persistencia, ya que ya se contaba con experiencia previa con esta plataforma y 
se ha optado por la simplicidad y velocidad de integración, la escalabilidad y la seguridad integrada que ofrece, además del soporte de \textit{Expo} y \textit{React Native}.
