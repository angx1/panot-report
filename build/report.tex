\documentclass[%
    school=etsisi,%
    type=pfg,%
    degree=61CI,%
]{upm-report}

%%%%%%%%%%%%%%%%%%%%%%%%%%%%%%%%%%%%%%%%%%%%%%%%%%%%%%%%%%%%%%%%%%%%%%%%
% TÍTULO, AUTORES Y DIRECTORES
%
\title{PANOT: Plataforma Móvil para la Gestión de la Inteligencia Relacional mediante Captura Asistida de Interacciones}
\author{Ángel Rodríguez Morán}
\bibauthor{}
\director{Elvira Amador Domínguez}

%%%%%%%%%%%%%%%%%%%%%%%%%%%%%%%%%%%%%%%%%%%%%%%%%%%%%%%%%%%%%%%%%%%%%%%%
% RESUMEN Y ABSTRACT
%
\abstract{spanish}{
    El presente Proyecto Fin de Grado (PFG) tiene como objetivo diseñar,
    desarrollar y desplegar PANOT, una plataforma móvil para la gestión de Inteligencia Relacional
    a través de la captura asistida de interacciones y operando bajo el paradigma de Experiencia Agéntica (AX).

    El objetivo de PANOT es facilitar el enriquecimiento progresivo de la cartera de contactos del usuario,
    mediante la captura de interacciones a través de una interfaz de entrada multimodal (voz y/o texto).
    Este proceso transforma los contactos, tradicionalmente estáticos, en entidades dinámicas que evolucionan
    para reflejar con mayor fidelidad el estado contextual de la relación.

    [continuar mas adelante después de terminar el proyecto...]

}
\keywords{spanish}{
    
}

\abstract{english}{
    [Write here the project summary in English]
}
\keywords{english}{
}

% Opcional
\reportquotation{
    [Cita opcional para el proyecto]
}{[Autor de la cita]}

% Opcional
\acknowledgements{
    [Escribir aquí los agradecimientos del proyecto]
}

%%%%%%%%%%%%%%%%%%%%%%%%%%%%%%%%%%%%%%%%%%%%%%%%%%%%%%%%%%%%%%%%%%%%%%%%
% GLOSARIO Y ABREVIATURAS
%
% Añadir aquí las entradas del glosario y acrónimos necesarios para el proyecto

\begin{document}

%%%%%%%%%%%%%%%%%%%%%%%%%%%%%%%%%%%%%%%%%%%%%%%%%%%%%%%%%%%%%%%%%%%%%%%%
% CAPÍTULOS

\chapter{Introducción}
\label{ch:introduccion}

%%%%%%%%%%%%%%%%%%%%%%%%%%%%%%%%%%%
% MOTIVACIÓN

\section{Motivación}

[Escribir aquí la motivación del proyecto]

%%%%%%%%%%%%%%%%%%%%%%%%%%%%%%%%%%%
% DESCRIPCIÓN Y ALCANCE DEL PROYECTO

\section{Descripción y Alcance del Proyecto}
\label{s:descripcion-alcance}

[Escribir aquí la descripción y alcance del proyecto]

%%%%%%%%%%%%%%%%%%%%%%%%%%%%%%%%%%%
% OBJETIVOS

\section{Objetivos}

[Escribir aquí los objetivos del proyecto]

%%%%%%%%%%%%%%%%%%%%%%%%%%%%%%%%%%%
% ESTRUCTURA DE LA MEMORIA

\section{Estructura de la memoria}

[Escribir aquí la estructura de la memoria]


\chapter{Estado del Arte y Fundamentos Tecnológicos}
\label{ch:estado-del-arte-y-fundamentos-tecnologicos}

%%%%%%%%%%%%%%%%%%%%%%%%%%%%%%%%%%%
% GESTIÓN DE LA INTELIGENCIA RELACIONAL: MÁS ALLÁ DEL CRM

\section{Gestión de la Inteligencia Relacional: Más Allá del CRM}

[Escribir aqui la gestion de la inteligencia relacional: mas alla del CRM]

%%%%%%%%%%%%%%%%%%%%%%%%%%%%%%%%%%%
% PARADIGMAS DE DISEÑO DE PRODUCTO EN LA ERA DE LA IA

\section{Paradigmas de Diseño de Producto en la Era de la IA}

[Escribir aqui los paradigmas de diseno de producto en la era de la IA]

%%%%%%%%%%%%%%%%%%%%%%%%%%%%%%%%%%%
% TECNOLOGÍAS PARA EL DESARROLLO DE APLICACIONES NATIVAS iOS

\section{Tecnologías para el Desarrollo de Aplicaciones Nativas iOS}

[Escribir aqui las tecnologías para el desarrollo de aplicaciones nativas iOS]

%%%%%%%%%%%%%%%%%%%%%%%%%%%%%%%%%%%
% ANÁLISIS COMPARATIVO

\section{Análisis Comparativo}

[Escribir aqui el analisis comparativo]


\chapter{Desarrollo del Proyecto}
\label{ch:desarrollo-del-proyecto}

%%%%%%%%%%%%%%%%%%%%%%%%%%%%%%%%%%%
% METODOLOGÍA Y ENTORNO DE DESARROLLO

\section{Metodología y Entorno de Desarrollo}

[Escribir aquí la metodología y entorno de desarrollo del proyecto]

%%%%%%%%%

\subsection{Gestión Ágil del Proyecto con GitHub Projects (Git Flow + Kanban)}

[Escribir aquí la gestión ágil del proyecto con GitHub Projects (Git Flow + Kanban)]

%%%%%%%%%

\subsection{Estrategia de Ramificación y Control de Versiones}

[Escribir aqui la estrategia de ramificacion y control de versiones]

%%%%%%%%%

\subsection{Estructura de Repositorios en la Organización}

[Escribir aqui la estructura de repositorios en la organizacion]

%%%%%%%%%%%%%%%%%%%%%%%%%%%%%%%%%%%
% ESPECIFICACIÓN DE REQUISITOS DE SOFTWARE

\section{Especificación de Requisitos de Software}

[Escribir aqui la especificacion de requisitos de software]

%%%%%%%%%%%%%%%%%%%%%%%%%%%%%%%%%%%
% DISEÑO DE LA ARQUITECTURA DEL SISTEMA

\section{Diseño de la Arquitectura del Sistema}

[Escribir aqui el diseño de la arquitectura del sistema]

%%%%%%%%%%%%%%%%%%%%%%%%%%%%%%%%%%%
% FASES DE LA IMPLEMENTACIÓN

\section{Fases de la Implementación}

[Escribir aqui las fases de la implementacion]

%%%%%%%%%%%%%%%%%%%%%%%%%%%%%%%%%%%
% DESPLIEGUE Y LANZAMIENTO DE PANOT

\section{Despliegue y Lanzamiento de PANOT}

[Escribir aqui el despliegue y lanzamiento de PANOT]


\chapter{Verificación y Resultados}
\label{ch:resultados-y-evaluacion}

[Escribir aquí los resultados y evaluación del proyecto]

\chapter{Conclusiones y Trabajo Futuro}
\label{ch:conclusiones-y-trabajo-futuro}

%%%%%%%%%%%%%%%%%%%%%%%%%%%%%%%%%%%
% CONCLUSIONES GENERALES

\section{Conclusiones Generales}

[Escribir aquí las conclusiones generales]

%%%%%%%%%%%%%%%%%%%%%%%%%%%%%%%%%%%
% APLICACIÓN DE CONOCIMIENTOS ADQUIRIDOS EN EL GRADO

\section{Aplicación de Conocimientos Adquiridos en el Grado}

[Escribir aquí la aplicación de conocimientos adquiridos en el grado]

%%%%%%%%%%%%%%%%%%%%%%%%%%%%%%%%%%%
% LÍNEAS DE TRABAJO FUTURO

\section{Líneas de Trabajo Futuro}

[Escribir aquí las líneas de trabajo futuro]



\end{document}
