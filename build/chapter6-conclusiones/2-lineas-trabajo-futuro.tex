\section{Líneas de Trabajo Futuro}
La publicación del \acrshort{mvp} marca el inicio de una etapa de recolección de métricas y \textit{feedback} cualitativo que permitirá pivotar o perseverar en la propuesta de valor. De 
cara a la evolución tecnológica del sistema, se proponen las siguientes líneas de trabajo:

En primer lugar, resulta imperativo alcanzar la paridad de plataformas mediante el desarrollo de la versión para Android. Aunque la lógica de negocio y la gestión de estado con 
\textit{Legend-State} son compatibles, la migración requerirá el desarrollo del módulo nativo de grabación y transcripción fuera del ecosistema de \textit{Apple Speech}, 
permitiendo así ampliar el embudo de captación de usuarios y obtener una muestra de datos más diversa para la fase de aprendizaje.

En segundo lugar, se plantea la transición hacia una IA descentralizada y privada. Se investigará la integración de modelos de lenguaje locales (\textit{On-device LLMs}), 
aprovechando tecnologías emergentes como \textit{Apple Intelligence}. El objetivo es permitir que la inferencia semántica ocurra íntegramente en el hardware del usuario, 
eliminando los costes operativos de \acrshort{api} externas y permitiendo un modelo de uso totalmente gratuito y \textit{offline}. Como paso intermedio, se contempla permitir 
que el usuario configure su propia \textit{API Key} de forma segura, delegando la elección del modelo y el control del gasto al propio usuario, culminando así un modelo de sistema totalmente privado por diseño.

Por último, se prevé una rearquitectura de la infraestructura de backend. Para eliminar el \textit{vendor lock-in} asociado actualmente a Supabase y mejorar la escalabilidad 
horizontal del sistema, se propone migrar hacia una solución basada en microservicios propietarios desplegados en contenedores. Esta evolución permitiría un control total sobre 
el motor de grafos y las colas de procesamiento, facilitando la integración de nuevas capacidades de análisis proactivo sin las limitaciones de los entornos de 
ejecución \textit{serverless} actuales.

Se propone el la siguiente arquitectura a modo de base para la posible nueva infraestructura:

[inlcuir esquema]