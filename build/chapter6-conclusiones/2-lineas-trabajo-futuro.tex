\section{Líneas de Trabajo Futuro}
La publicación del \acrshort{mvp} marca el inicio de la etapa de recolección de métricas y \textit{feedback} cualitativo de usuarios reales que permitirá pivotar o perseverar en la propuesta de valor. De 
cara a la evolución tecnológica del sistema, se proponen las siguientes líneas de trabajo y mejoras técnicas:

En primer lugar, el paso que resulta más natural, es alcanzar la paridad en sistemas operativos móviles mediante el desarrollo de la versión para Android. Aunque la lógica de negocio y la gestión de estado con 
\textit{Legend-State} son compatibles, la implementación hacia esta nueva plataforma requerirá el desarrollo de un módulo nativo de android de grabación y transcripción fuera del ecosistema iOS, 
permitiendo así ampliar el embudo de captación de usuarios y obtener una muestra de datos más diversa para la fase de aprendizaje.

En segundo lugar, se plantea la transición hacia un sistema de procesamiento semántico descentralizado y privado. Se investigará la integración de modelos de lenguaje locales (\textit{On-device LLMs}), 
aprovechando tecnologías emergentes como los modelos fundacionales de \textit{Apple Intelligence}. El objetivo es permitir que la inferencia semántica ocurra íntegramente en el hardware del usuario, 
eliminando los costes operativos que acarrean el uso de proveedores externos y permitiendo un modelo de uso totalmente gratuito y offline. Como paso intermedio, se contempla permitir 
que el usuario configure su propia \textit{API Key} de forma segura, delegando la elección del modelo y el control del gasto al propio usuario, culminando así un modelo de sistema totalmente privado por diseño.

Por último, para eliminar el \textit{vendor lock-in} asociado actualmente a Supabase y mejorar la escalabilidad 
horizontal del sistema, se propone migrar hacia una solución similar pero basada en microservicios propios desplegados en contenedores, tomando la siguiente Figura (\ref{fig:posible-infra}), 
como una posible nueva arquitectura:

\begin{figure}[ht!]
    \centering
    \includegraphics[width=.9\textwidth]{figures/posible-infra.png}
    \caption{Arquitectura propuesta para backend propio de PANOT.}
    \label{fig:posible-infra}
\end{figure}



