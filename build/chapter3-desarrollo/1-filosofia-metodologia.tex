%%%%%%%%%%%%%%%%%%%%%%%%%%%%%%%%%%%
% METODOLOGÍA Y ENTORNO DE DESARROLLO



\section{Filosofía y Metodología de Desarrollo}

El desarrollo de PANOT se ha planteado desde una perspectiva que prioriza la validación empírica y la eficiencia en 
la entrega de valor, alejándose de los enfoques tradicionales de planificación predictiva en cascada. Dado el carácter 
innovador de la propuesta —la gestión de la inteligencia relacional mediante IA—, el proyecto se enfrenta a un alto 
grado de incertidumbre, no tanto en la viabilidad técnica, sino en la validación del producto por parte del usuario final.

Para mitigar este riesgo, se ha adoptado una filosofía fundamentada en los principios de \textit{Lean Startup} \cite{ries2011lean}
y \textit{Lean Thinking} \cite{2012leanthinking}. Bajo este prisma, el desarrollo de software no se entiende como la ejecución de una especificación 
cerrada, sino como un proceso de descubrimiento orientado a minimizar el desperdicio (Muda) —entendido como cualquier 
esfuerzo o característica que no genere valor para el usuario.

En consecuencia, la metodología de trabajo implementada en este Trabajo Fin de Grado no tiene como objetivo la 
finalización de un producto comercial definitivo, sino la construcción y despliegue de un \acrlong{mvp}. Este \acrshort{mvp} 
constituye el punto de partida necesario para ejecutar la primera fase del ciclo \gls{construir-medir-aprender}, 
permitiendo someter a prueba las hipótesis conceptuales detalladas en el capítulo anterior y generar un aprendizaje que 
guíe la evolución futura de la plataforma. 


\subsection{Enfoque Lean}

[TFG representa la fase de "Construir" (Build) dentro del ciclo Construir-Medir-Aprender y no un desarrollo en cascada tradicional]

\subsection{Metodología y Flujo de Desarrollo}

[Explicar metodología GitFlow, sistema de ramas, organización de github, proyecto de github con tablero kanban
y el flujo con ejemplo de la implementación]
