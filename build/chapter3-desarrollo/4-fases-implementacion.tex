\clearpage
\section{Implementación del Sistema}

Como se mencionó en la sección \ref{sec:enfoque-lean}, el desarrollo de PANOT se centra en la producción de un artefacto que permita 
validar las hipótesis de valor. Es por esto que, a diferencia de un modelo incremental tradicional, la implementación de este sistema 
se ha llevado a cabo como un esfuerzo unificado para materializar el \acrshort{mvp}. Bajo esta premisa, la implementación de PANOT se ha 
organizado en ejes de desarrollo concurrentes para maximizar la velocidad de desarrollo y permitir el trabajo paralelo sobre las diferentes 
capacidades del sistema. 

Es esencial señalar que en este proyecto, el despliegue no es una fase posterior de la implementación, sino su culminación. La publicación 
de la aplicación \textit{"Semilla"} en una plataforma de distribución es el evento de cierre del proceso de construcción (Fase \textit{Construir}) y da comienzo 
a la fase de medición (Fase \textit{Medir}). Por ello, la implementación incluirá tareas como la instrumentación de herramientas de medición ya que, sin ellas, 
el \acrshort{mvp} no podría cumplir su propósito de validación de la hipótesis de valor.

\subsection{Organización \texttt{panot-hq}}

Para facilitar la gestión y organización del código del proyecto, se ha creado la organización \texttt{panot-hq} en \textit{GitHub}. 
Esta organización centraliza todos los repositorios de desarrollo de PANOT, permitiendo una estructura clara y escalable del proyecto. Además, como ya se explicó en 
la sección \ref{sec:metodologia-flujo-desarrollo}, se ha configurado un proyecto dentro de la organización para agrupar las unidades de trabajo y el progreso de las mismas.
Esta estructura se ha establecido para, además de los beneficios obvios de trazabilidad y seguimiento, permitir la incorporación 
de nuevos colaboradores en el futuro. Los repositorios levantados de la organización son los siguientes:

\vspace{.5cm}
\begin{figure}[ht!]
    \centering
    \includegraphics[width=0.8\textwidth]{figures/panot-hq-repos.png}
    \caption{Repositorios de la organización \texttt{panot-hq}}
    \label{fig:panot-hq-repos}
\end{figure}

{\small
\begin{itemize}
    \item \texttt{panot-hq/mobile}: Código de la aplicación móvil.
    \item \texttt{panot-hq/landing}: Código de [incluir en el glosario def de landing] landing de PANOT.
    \item \texttt{panot-hq/edge-backend}: Código de las Edge Functions que se despliegan en Supabase.
    \item \texttt{panot-hq/graph-visualizer}: Código del entorno de pruebas para la visualización de grafos de conocimiento.
    \item \texttt{panot-hq/panot-speech}: Código del paquete \textit{npm} del módulo nativo iOS para transcripción en tiempo real para dispositivos iOS.
\end{itemize}
}

\subsection{Ejes de Desarrollo}

Se define eje de desarrollo como la línea de trabajo dedicada al desarrollo de una capacidad del sistema, incluyendo su integración con el 
resto de los componentes con los que interactúe. En el caso de PANOT, se han llevado a cabo los siguientes ejes de desarrollo:


\begin{table}[ht!]
\scriptsize
\begin{tabularx}{\textwidth}{|l|X|X|}
\hline
\rowcolor[HTML]{EFEFEF} 
\textit{Eje}                                                   & \textit{Propósito}                                                             & \textit{Repositorios Involucrados} \\ \hline
(1) Aplicación Móbil                                           & Desarrollo de la UI y componentes de la aplicación de PANOT                    & \texttt{panot-hq/mobile} \newline \texttt{panot-hq/panot-speech}                                  \\ \hline
(2) Base de Datos                                              & Desarrollo y despliegue de infraestructura de datos       & \texttt{panot-hq/mobile}                                  \\ \hline
(3) Sistema Multi-Agente                                       & Desarrollo de sistema de procesamiento de información                 & \texttt{panot-hq/edge-backend} \newline \texttt{panot-hq/graph-visualizer} \newline \texttt{panot-hq/mobile}                                 \\ \hline
(4) Sistema de Cola de Trabajos                                & Desarrollo de sistema de gestión de trabajadores                               & \texttt{panot-hq/edge-backend} \newline \texttt{panot-hq/mobile}                                  \\ \hline
(5) Landing de PANOT                                           & Desarrollo de página web                                              & \texttt{panot-hq/landing}                                  \\ \hline
\end{tabularx}
\caption{Ejes de desarrollo de PANOT}
\label{tab:ejes-trabajo}
\end{table}


\subsection{Línea de Tiempo}








