\clearpage
\section{Implementación del Sistema}

Como se mencionó en la sección \ref{sec:enfoque-lean}, el desarrollo de PANOT se centra en la producción de un artefacto que permita 
validar las hipótesis de valor. Es por esto que, a diferencia de un modelo incremental tradicional, la implementación de este sistema 
se ha llevado a cabo como un esfuerzo unificado para materializar el \acrshort{mvp}. Bajo esta premisa, la implementación de PANOT se ha 
organizado en ejes de desarrollo concurrentes para maximizar la velocidad de desarrollo y permitir el trabajo paralelo sobre las diferentes 
capacidades del sistema. 

Es esencial señalar que en este proyecto, el despliegue no es una fase posterior de la implementación, sino su culminación. La publicación 
de la aplicación \textit{"Semilla"} en una plataforma de distribución es el evento de cierre del proceso de construcción (Fase \textit{Construir}) y da comienzo 
a la fase de medición (Fase \textit{Medir}). Por ello, la implementación incluirá tareas como la instrumentación de herramientas de medición ya que, sin ellas, 
el \acrshort{mvp} no podría cumplir su propósito de validación de la hipótesis de valor.

\subsection{Organización \texttt{panot-hq}}

Para facilitar la gestión y organización del código del proyecto, se ha creado la organización \texttt{panot-hq} en \textit{GitHub}. 
Esta organización centraliza todos los repositorios de desarrollo de PANOT, permitiendo una estructura clara y escalable del proyecto. Además, como ya se explicó en 
la sección \ref{sec:metodologia-flujo-desarrollo}, se ha configurado un proyecto dentro de la organización para agrupar las unidades de trabajo y el progreso de las mismas.
Esta estructura se ha establecido para, además de los beneficios obvios de trazabilidad y seguimiento, permitir la incorporación 
de nuevos colaboradores en el futuro. Los repositorios levantados de la organización son los siguientes:

\vspace{.5cm}
\begin{figure}[ht!]
    \centering
    \includegraphics[width=0.9\textwidth]{figures/panot-hq-repos.png}
    \caption{Repositorios de la organización \texttt{panot-hq}}
    \label{fig:panot-hq-repos}
\end{figure}

{\small
\begin{itemize}
    \item \texttt{panot-hq/mobile}: Código de la aplicación móvil.
    \item \texttt{panot-hq/landing}: Código de [incluir en el glosario def de landing] landing de PANOT.
    \item \texttt{panot-hq/edge-backend}: Código de las Edge Functions que se despliegan en Supabase.
    \item \texttt{panot-hq/graph-visualizer}: Código del entorno de pruebas para la visualización de grafos de conocimiento.
    \item \texttt{panot-hq/panot-speech}: Código del paquete \textit{npm} del módulo nativo iOS para transcripción en tiempo real para dispositivos iOS.
\end{itemize}
}

\subsection{Ejes de Desarrollo}

Se define eje de desarrollo como la línea de trabajo dedicada al desarrollo de una capacidad del sistema, incluyendo su integración con el 
resto de los componentes con los que interactúe. En el caso de PANOT, se han llevado a cabo los siguientes ejes de desarrollo:


\begin{table}[ht!]
\footnotesize
\begin{tabularx}{\textwidth}{|l|X|X|}
\hline
\rowcolor[HTML]{EFEFEF} 
\textit{Eje} & \textit{Repositorios Involucrados} & \textit{Tecnologías Utilizadas} \\ \hline
(1) Aplicación Móvil & {\scriptsize \texttt{panot-hq/mobile} \newline \texttt{panot-hq/panot-speech}} & Expo, SwiftUI, TypeScript, PostHog, Stripe \\ \hline
(2) Base de Datos & {\scriptsize \texttt{panot-hq/mobile}} & Supabase, PostgreSQL, TypeScript \\ \hline
(3) Sistema Multi-Agente & {\scriptsize \texttt{panot-hq/edge-backend} \newline \texttt{panot-hq/graph-visualizer} \newline \texttt{panot-hq/mobile}} & Supabase, Deno, TypeScript, LangChain, LangSmith \\ \hline
(4) Sistema de Cola de Trabajos & {\scriptsize \texttt{panot-hq/edge-backend} \newline \texttt{panot-hq/mobile}} & Supabase, Deno, TypeScript \\ \hline
(5) Landing de PANOT & {\scriptsize \texttt{panot-hq/landing}} & Next.js, TypeScript \\ \hline
\end{tabularx}
\caption{Ejes de desarrollo de PANOT}
\label{tab:ejes-trabajo}
\end{table}

{\small
\begin{enumerate}
    \item[(1)] \textbf{Aplicación Móvil}: Desarrollo de la UI y componentes de la aplicación de PANOT.
    \item[(2)] \textbf{Base de Datos}: Desarrollo y despliegue de la infraestructura de datos.
    \item[(3)] \textbf{Sistema Multi-Agente}: Desarrollo de sistema de procesamiento de información.
    \item[(4)] \textbf{Sistema de Cola de Trabajos}: Desarrollo de sistema de gestión de trabajadores.
    \item[(5)] \textbf{Landing de PANOT}: Desarrollo de la página web.
\end{enumerate}
}


\subsection{Evolución Cronológica y Concurrente}

A continuación, se detalla la línea de tiempo del desarrollo, organizada por hitos mensuales. 

\subsubsection{Octubre 2025}

El primer mes de desarrollo se centró en establecer la autenticación básicadel usuario y la capacidad de gestión de contactos e interacciones local. El objetivo 
fue alcanzar una aplicación funcional en modo \textit{offline} que permitiera la captura de datos manual y por voz. Durante este mes se trabajó principalmente 
en los ejes de desarrollo (1) \textit{Aplicación Móvil} y (2) \textit{Base de Datos}, estableciendo la infraestructura de datos en Supabase y desarrollando 
las interfaces de usuario en {\footnotesize \texttt{panot-hq/mobile}}.

\vspace{.5cm}
\begin{figure}[ht!]
    \centering
    \includegraphics[width=0.7\textwidth]{figures/closed-issues-octubre.png}
    \caption{Issues cerradas en octubre de 2025}
    \label{fig:closed-issues-octubre}
\end{figure}

{\small

\paragraph{Gestión de Usuario (\hyperref[req:FR-01]{FR-01})} Se crearon las tablas necesarias en Supabase para la gestión de usuarios y 
se desarrolló en {\footnotesize \texttt{panot-hq/mobile}} la interfaz para conectarse con Supabase. Además, se implementó el flujo de autenticación 
mediante Email \acrshort{otp} y Apple ID (Issue \#22). Esto permitió que el resto del desarrollo contara con un contexto de seguridad 
basado en el ID de usuario.

\paragraph{Gestión de Contactos (\hyperref[req:FR-04]{FR-04}, \hyperref[req:FR-08]{FR-08})} Se crearon las tablas necesarias en Supabase 
para la gestión de contactos y se desarrolló en {\footnotesize \texttt{panot-hq/mobile}} la interfaz para conectarse con Supabase. Se completó el núcleo 
del \acrshort{crud} manual de contactos (Issues \#6, \#7, \#8). Por último, se integró la capa de persistencia local (\hyperref[req:RNF-DISP-004]{RNF-DISP-004}) 
utilizando \textit{Legend-State} y \textit{AsyncStorage} (Issue \#33).

\paragraph{Gestión de Interacciones (\hyperref[req:FR-10]{FR-10}, \hyperref[req:FR-11]{FR-11}, \hyperref[req:FR-12]{FR-12})} Se crearon las tablas necesarias 
en Supabase para la gestión de interacciones además de la interfaz de interacción en {\footnotesize \texttt{panot-hq/mobile}} y se desarrolló el paquete nativo {\footnotesize \texttt{panot-speech}} para habilitar la grabación de voz y 
la transcripción inmediata (Issues \#12, \#26) y se integró en la aplicación móvil.

\paragraph{Instrumentación Base} Se implementó la lógica de permisos contextuales (\textit{micrófono y contactos}) necesaria 
para el cumplimiento normativo y funcional de Apple (Issue \#25).
}

\vspace{.7cm}
\begin{figure}[ht!]
    \centering
    \includegraphics[width=1\textwidth]{figures/git-commit-tree-octubre.png}
    \caption{Árbol de commits de octubre (y finales de septiembre) de 2025 en repositorio Git local de \texttt{panot-hq/mobile}}
    \label{fig:git-commit-tree-octubre}
\end{figure}

\subsubsection{Noviembre 2025}

Noviembre se centró en completar las funcionalidades restantes de gestión de contactos además de implementar las capacidades de 
inteligencia relacional básicas. El objetivo fue alcanzar un sistema funcional que permitiera la actualización automática de los detalles  
de los contactos mediante el procesamiento de interacciones. Durante este mes se continuó trabajando en el eje (1) \textit{Aplicación Móvil} 
y se inició el desarrollo del eje (3) \textit{Sistema Multi-Agente} con la implementación de la lógica básica de actualización automática 
del contexto de contactos.

\vspace{.5cm}
\begin{figure}[ht!]
    \centering
    \includegraphics[width=0.7\textwidth]{figures/closed-issues-noviembre.png}
    \caption{Issues cerradas en noviembre de 2025}
    \label{fig:closed-issues-noviembre}
\end{figure}

{\small

\paragraph{Completar Gestión de Contactos (\hyperref[req:FR-05]{FR-05}, \hyperref[req:FR-07]{FR-07}, \hyperref[req:FR-09]{FR-09})} Se cerró 
el trabajo de las operaciones del \acrshort{crud} de contactos: se implementó 
la funcionalidad para importar contactos desde la agenda nativa del dispositivo (Issue \#10), la búsqueda de contactos con filtrado por nombre 
y palabras clave (Issue \#11), y la eliminación permanente de contactos (Issue \#9).

\paragraph{Captura de Voz para Contactos (\hyperref[req:FR-06]{FR-06})} Se integró la captura de voz para crear contactos mediante 
dictado en lenguaje natural y se desarrollaron mejoras en la visualización de audio (Issue \#37, PRs \#38, \#40).

\paragraph{Configuraciones de la Aplicación (\hyperref[req:FR-16]{FR-16}, \hyperref[req:FR-17]{FR-17})} Se añadió la pantalla de 
ajustes de la aplicación con configuraciones como el idioma de transcripción (Issue \#42, PRs \#44, \#45). 
Esta pantalla estableció la base para futuras capacidades de ajustes y el apartado de feedback.

\paragraph{Finalización de Autenticación (\hyperref[req:FR-01]{FR-01})} Se completó el flujo de inicio de sesión mediante Email y 
Apple ID, consolidando la implementación iniciada en octubre (Issue \#23, PR \#46). Esto permitió que el sistema contara con un 
mecanismo de autenticación robusto y funcional tanto para el \textit{sign-up} (registro) como para el \textit{log-in} (inicio de sesión).

\paragraph{Inteligencia Relacional Básica (\hyperref[req:FR-14]{FR-14})} Se implementó la lógica para actualizar automáticamente 
los detalles de un contacto cuando se registra una nueva interacción (Issue \#48, PR \#49). Esta funcionalidad 
constituyó el primer paso hacia el sistema de inteligencia relacional completo. Como dato importante, en este punto del proyecto, 
todavía no se contaba con el sistema multi-agente ni la clase de grafo semántico, es decir, todavía no existía un sistema de procesamiento relacional 
ni en el procesamiento de interacciones ni en la creación de contactos "hablando sobre ellos".

}

\vspace{.7cm}
\begin{figure}[ht!]
    \centering
    \includegraphics[width=1\textwidth]{figures/git-commit-tree-noviembre.png}
    \caption{Árbol de commits de noviembre de 2025 en repositorio Git local de \texttt{panot-hq/mobile}}
    \label{fig:git-commit-tree-noviembre}
\end{figure}


\newpage
\subsubsection{Diciembre 2025}

Diciembre se centró en completar las funcionalidades esenciales para el lanzamiento del \acrshort{mvp}, incluyendo la monetización, 
el sistema multi-agente, el sistema de cola de procesamiento, la capa de observabilidad y los mecanismos de soporte. El objetivo fue alcanzar 
un sistema listo para despliegue. Durante este mes se trabajó principalmente en los ejes (1) \textit{Aplicación Móvil}, (2) \textit{Base de Datos}, 
(3) \textit{Sistema Multi-Agente} y (4) \textit{Sistema de Cola de Trabajos}.

\vspace{.5cm}
\begin{figure}[ht!]
    \centering
    \includegraphics[width=0.7\textwidth]{figures/closed-issues-diciembre.png}
    \caption{Issues cerradas en diciembre de 2025}
    \label{fig:closed-issues-diciembre}
\end{figure}

{\small

\paragraph{Gestión de Pagos y Suscripciones} Se completó la lógica del \textit{paywall} y la integración con \textit{Stripe} para suscripciones mensuales 
y acceso a funcionalidades premium (Issue \#50) con el desarrollo de la Edge Function \texttt{stripe-infra}. Esta implementación permitió establecer el 
modelo de monetización del sistema y controlar el acceso a las capacidades de procesamiento relacional de la aplicación.

\paragraph{Sistema Multi-Agente (\hyperref[req:FR-14]{FR-14})} Se implementó el sistema multi-agente \ref{fig:diagrama-multi-agente} completo para el procesamiento de información, incluyendo 
la aceptación y actualización de datos derivados en contactos (Issue \#52). Los sub-hitos principales fueron:
\begin{enumerate}
    \item Desarrollo de {\footnotesize \texttt{relational-agent}} en {\footnotesize \texttt{panot-hq/edge-backend}} utilizando \textit{LangChain} 
    para la orquestación de agentes e integración con la aplicación móvil.
    \item Creación de las tablas necesarias en Supabase para los nodos y las aristas semánticas.
    \item Desarrollo de la herramienta {\footnotesize \texttt{panot-hq/graph-visualizer}} para facilitar el desarrollo, pruebas 
    y validación de la estructura del grafo semántico del usuario.
    \item Integración de \textit{LangSmith} como capa de observabilidad del sistema agéntico, permitiendo así monitorizar el flujo de procesamiento con información como la latencia, los 
    tokens utilizados o el coste de la ejecución de cada llamada al sistema.
\end{enumerate}

\vspace{.5cm}
\begin{figure}[ht!]
    \centering
    \includegraphics[width=1\textwidth]{figures/ejemplo-traza-langsmith.png}
    \caption{Ejemplo de traza de ejecución en LangSmith del sistema multi-agentede PANOT}
    \label{fig:ejemplo-traza-langsmith}
\end{figure}
\vspace{.5cm}

\paragraph{Sistema de Cola de Procesos (\hyperref[req:FR-14]{FR-14})} Se implementó el sistema de cola de procesos para la gestión asíncrona entre 
usuarios del procesamiento de información \ref{fig:processing-system-activity-diagram}. Los sub-hitos principales fueron:
\begin{enumerate}
    \item Desarrollo de {\footnotesize \texttt{worker-infra}} en {\footnotesize \texttt{panot-hq/edge-backend}} y su integración con la aplicación móvil.
    \item Creación de las tablas necesarias en Supabase para la cola de procesos y los \textit{workers} de los usuarios.
\end{enumerate}

\paragraph{Capa de Observabilidad (\hyperref[req:RNF-MANT-008]{RNF-MANT-008})} Se instrumentó la aplicación para el análisis de comportamiento de usuarios  
utilizando \textit{PostHog} (Epic \#55). Esta implementación permitió la captura de métricas para validar las hipótesis de valor definidas en el proyecto.

\paragraph{Sistema de Soporte (\hyperref[req:FR-16]{FR-16})} Se añadió el mecanismo de soporte y recolección de \textit{tickets} para reportes 
de errores o solicitudes de funcionalidades desde la aplicación (Issue \#43). Este mecanismo estableció el canal de comunicación entre 
los usuarios y el equipo de desarrollo, permitiendo la recolección de feedback estructurado.
}

\vspace{.7cm}
\begin{figure}[ht!]
    \centering
    \includegraphics[width=1\textwidth]{figures/git-commit-tree-diciembre.png}
    \caption{Árbol de commits de diciembre de 2025 (y principios de enero de 2026) en repositorio Git local de \texttt{panot-hq/mobile}}
    \label{fig:git-commit-tree-diciembre}
\end{figure}

\subsubsection{Enero 2026}

Habiendo finalizado el desarrollo completo del \acrshort{mvp}, principios de enero (del 6 al 8) de 2026 se dedicaron al desarrollo de la 
página web landing de PANOT tanto en inglés como en español. En la web se han incluido las siguientes secciones:

{\small
\begin{itemize}
    \item \textbf{Inicio}: Página de inicio de la web donde se expone la hipótesis de valor y el link de descarga de la aplicación.
    \item \textbf{Changelog}: Línea de tiempo de cambios de PANOT.
    \item \textbf{Política de Privacidad}: Política de privacidad de PANOT. Necesaria para la publicación de la aplicación en los canales de distribuciónd de Apple.
    \item \textbf{Términos y Condiciones}: Términos y condiciones de uso de PANOT.
\end{itemize}
}
Esto engloba el eje de desarrollo (5) \textit{Landing de PANOT} en el repositorio {\footnotesize \texttt{panot-hq/landing}}.


\subsubsection{Resumen}

Como se ha podido comprobar, tanto la aplicación móvil como la infraestructura del backend y la landing del \acrshort{mvp} se han llebado 
a cabo en un periodo de aproximadamente 4 meses, desde octubre de 2025 a enero de 2026.

\vspace{.5cm}
\begin{figure}[ht!]
    \centering
    \includegraphics[width=1\textwidth]{figures/burn-up-lines-stacked.png}
    \caption{Gráfico de burndup de issues cerrados en el proyecto de PANOT}
    \label{fig:timeline-development}
\end{figure}

En esta línea de tiempo, se han descartado cambios que se han implementado después de estas fechas ya que engloban cambios no críticos como 
correcciones de errores menores o iteraciones en la interfaz de usuario. 

