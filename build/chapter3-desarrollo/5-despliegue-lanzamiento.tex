\clearpage
\section{Despliegue y Lanzamiento de PANOT}

Tras culminar con las fases de análisis, diseño e implementación, el proyecto entra en su fase clave de puesta en producción.
Este proceso no se limita únicamente a transferir el código a un servidor, sino que engloba una serie de verificaciones de seguridad, 
la preparación de artefactos nativos y la gestión administrativa ante las plataformas de distribución (en este caso la App Store de Apple).
El objetivo final es garantizar que el \acrshort{mvp} sea accesible para los usuarios finales bajo los estándares de seguridad, rendimiento 
y diseño definidos en los capítulos anteriores \ref{tab:non-functional-requirements} y los impuestos por las políticas de Apple.

Dada la naturaleza académica de este proyecto y el marco normativo de la Universidad Politécnica de Madrid, el despliegue está orientado a la 
materialización de un artefacto funcional ajustando ciertas capacidades comerciales para alinearse con los derechos de explotación de la institución.
[VER BIEN ESTO Y MODIFICARLO SI NO ES CORRECTO]

\subsection{Preparación del Entorno y Adaptación Académica}

La transición hacia una versión publicable ha requerido una configuración específica de la infraestructura y la aplicación móvil de PANOT. A diferencia 
de una versión comercial abierta, y con el objetivo de garantizar la sostenibilidad del proyecto a futuro, se ha desplegado una versión 
que prioriza las capacidades de almacenamiento local y gestión de usuarios, contactos e interacciones descritas en la especificación del sistema. En este sentido, 
se ha optado por una configuración que prescinde de los módulos de pago recurrente y de las llamadas activas al sistema multi-agente en el entorno desplegado. 
Esta decisión técnica permite que tanto el evaluador y como futuros usuarios interactúen con una aplicación estable y fluida, centrada en la experiencia 
de usuario y la arquitectura local-first, sin riesgos operativos o económicos.

En cuanto al cliente móvil, el proceso de empaquetado se realizó de forma manual utilizando \textit{Xcode}. Este flujo de trabajo se inicia con la 
generación del proyecto nativo de iOS mediante el comando de pre-construcción ({\footnotesize \texttt{npx expo prebuild --platform ios}}) de Expo, integrando todos 
los módulos que componen el proyecto. Una vez dentro de \textit{Xcode}, se procedió a la firma del código mediante certificados de distribución oficiales, 
paso indispensable para generar el archivo con extensión {\footnotesize \texttt{.ipa}} (iOS App Store Package) denominado binario (\textit{binary}). Este archivo binario, constituye la entrega 
final y representaría el esfuerzo de ingeniería realizado listo para su distribución.

\subsection{Cumplimiento Normativo y Directrices de Apple}

La subida del binario a la plataforma \textit{App Store Connect} marca el inicio de una auditoría técnica y de contenido por parte de Apple. Para superar 
este proceso de revisión (App Review), el desarrollo de PANOT ha tenido que alinearse con las \textit{App Review Guidelines} \cite{apple-app-store-review-guidelines}, un conjunto de reglas 
estrictas que garantizan la seguridad y calidad de las aplicaciones en el ecosistema iOS.

Uno de los puntos críticos ha sido el cumplimiento de la directiva sobre Privacidad y Protección de Datos (\textit{Legal - 5.1 Privacy}). Dado que PANOT 
solicita acceso a información sensible como la agenda de contactos y el micrófono del dispositivo, se han tenido que implementar mensajes 
de propósito claros y concisos que expliquen al usuario exactamente por qué y para qué se requieren dichos permisos. Este compromiso técnico se extiende al rendimiento (\textit{Performance}), 
donde la arquitectura local-first y la gestión de estados con \textit{Legend-State} permiten cumplir con la exigencia de Apple de ofrecer una respuesta instantánea y 
eficiente, minimizando la latencia de red y optimizando el uso de los recursos del hardware. Paralelamente, el diseño de PANOT se ha trazado 
siguiendo las Human Interface Guidelines (\acrshort{hig}) \cite{apple-human-interface-guidelines} para asegurar una experiencia basada en los pilares de 
\textit{Jerarquía}, \textit{Armonía} y \textit{Consistencia}, donde además, el enfoque voice-first permite reducir la fricción cognitiva conforme a los 
estándares de usabilidad de la plataforma.

Finalmente, la decisión de omitir la pasarela de pagos en esta versión académica de PANOT simplifica el cumplimiento de las directrices 
de negocio (\textit{Business - 3.1 Payments}), evitando la complejidad de las compras integradas obligatorias y permitiendo que el proyecto se presente como un 
producto de software estable y gratuito. Esta aproximación valida el sistema bajo las normativas de seguridad y diseño del ecosistema de Apple, reduciendo simultáneamente 
la incertidumbre legal derivada de los derechos de explotación de la Universidad Politécnica de Madrid.

\subsection{Estrategia de Lanzamiento y Visibilidad}

La finalización de este proceso de despliegue se alcanza con la aprobación de la aplicación por parte del equipo de revisión de Apple. Este hito 
no solo representa un éxito a nivel administrativo, sino que actúa como una certificación de que el software cumple con los estándares de seguridad, rendimiento 
y diseño exigidos por Apple. La confirmación de entrada en la \textit{App Store} permite habilitar los siguientes canales de captación diseñados para atraer a los 
primeros usuarios y comenzar la fase de validación (\textit{Medir}).

[INLUIR CAPTURA DE APROBACIÓN DE LA APP REVIEW]

Con la aplicación disponible para el público general, la estrategia de difusión se apoya en la landing page de PANOT. Esta web, que fue indispensable durante el 
envío a revisión para alojar las políticas de privacidad exigidas por Apple, cumple el propósito de canalizar a los usuarios hacia la tienda oficial, reforzando la 
propuesta de valor de la Inteligencia Relacional y sirviendo como punto de referencia para aquellos que descubran el proyecto a través de canales externos.

Finalmente, el lanzamiento principal se ha planificado en la plataforma \textit{Product Hunt}. Al tratarse de un sistema con un alto componente de innovación, resulta el escenario 
más adecuado para cerrar el primer ciclo de la metodología \textit{Lean Startup}. La visibilidad de este canal no solo busca funcionar como embudo de captación de usuarios, sino 
que además, permite la recolección de métricas de comportamiento de usuarios reales que habilitan la validación de la hipótesis de valor del proyecto. De este modo, la 
publicación oficial en la App Store y la subsiguiente campaña de visibilidad en \textit{Product Hunt} cierran el bloque de desarrollo de la memoria.

[INCLUIR CAPTURA DE LAUNCH EN PRODUCT HUNT]