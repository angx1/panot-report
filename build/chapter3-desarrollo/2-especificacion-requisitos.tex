%%%%%%%%%%%%%%%%%%%%%%%%%%%%%%%%%%%
% ESPECIFICACIÓN DE REQUISITOS DE SOFTWARE

\section{Definición del PMV y Especificación de Requisitos}

Siguiendo el marco teórico expuesto por \textit{Eric Ries} \cite{ries2011lean} el desarrollo de PANOT se fundamenta en la identificación y validación de \textit{Asunciones de Salto Fe}.
Estas son las premisas de mayor incertidumbre y riesgo que, de resultar falsas, invalidarían la totalidad del proyecto.

Las dos asunciones fundamentales en este marco son:

{\small
\begin{itemize}
\setlength{\itemsep}{0pt}
    \setlength{\parsep}{0pt}
    \setlength{\topsep}{0pt}
    \setlength{\partopsep}{0pt}
    \item \textbf{Hipótesis de Valor:} Determina si un producto o servicio proporciona valor real a los usuarios una vez que lo utilizan. 
    Se valida analizando la retención y el uso voluntario y respondería a la pregunta: \textit{¿El usuario percibe suficiente utilidad en el 
    sistema como para incorporarlo a su vida diaria?}

    \item \textbf{Hipótesis de Crecimiento:} Determina cómo los nuevos usuarios descubrirán el producto o servicio. Se valida analizando 
    los motores de crecimiento (viral, pegajoso o remunerado). Responde a la pregunta: \textit{¿Existe un mecanismo sostenible para adquirir 
    nuevos usuarios y escalar el producto?}
\end{itemize}
}

A modo aclarativo, el esfuerzo de ingeniería de este Proyecto Fin de Grado se ha focalizado en la validación de la \textit{Hipótesis de Valor}, dejando
la validación de la \textit{Hipótesis de Crecimiento} para estados futuros del ciclo de vida de la aplicación.

\subsection{Hipótesis de Valor}

Teniendo en cuenta lo anterior, el artefacto para validar la \textit{Hipótesis de Valor} es un \gls{mvp} que representa la versión mínima funcional de PANOT. 
La hipótesis concreta es que los usuarios adoptarán el sistema porque resuelve la carga cognitiva y el esfuerzo manual que supone mantener relaciones 
personales. Esta hipótesis general se desglosa en tres premisas clave:

{\small
\begin{enumerate}
    \setlength{\itemsep}{0pt}
    \setlength{\parsep}{0pt}
    \setlength{\topsep}{0pt}
    \setlength{\partopsep}{0pt}
    \item \textbf{Valor de la Inmediatez:} Se asume que los usuarios serán constantes en el registro de información 
    solo si el esfuerzo requerido y la fricción son cercanos a cero. La captura por voz ofrece un valor superior a la entrada de texto tradicional al permitir 
    registrar interacciones en movimiento y sin necesidad de una atención visual exhaustiva.
    
    \item \textbf{Valor de la Mantenibilidad Pasiva:} Se asume que los usuarios perciben valor en un sistema que mantiene actualizada la información 
    de sus contactos (cambios de trabajo, mudanzas, gustos) de forma autónoma, liberándoles de la tarea administrativa de editar fichas 
    de contacto manualmente.
    
    \item \textbf{Valor de la Memoria Aumentada:} Se asume que los usuarios obtienen utilidad real al disponer de resúmenes contextuales 
    precisos antes de una interacción, mejorando así la calidad de sus relaciones sociales gracias a una "memoria externa" que procesa el contexto del contacto por ellos.
\end{enumerate}
}

\subsection{Requisitos Funcionales}

\subsection{Requisitos No Funcionales} 