\section{Objetivos del Proyecto}
\label{s:objetivos}

El objetivo general de este proyecto es diseñar, desarrollar y lanzar un Producto Mínimo Viable (\acrshort{mvp}) que sirva como instrumento técnico para validar las hipótesis de valor 
asociadas al propósito de PANOT. Para alcanzar este objetivo general, se han definido los siguientes objetivos específicos:

\begin{itemize}
    \item \textbf{O1}\label{o1}. Investigar y analizar las herramientas existentes para la gestión de relaciones personales.  
    \item \textbf{O2}\label{o2}. Diseñar una arquitectura de persistencia \textit{local-first} que garantice la disponibilidad total del sistema en entornos sin conectividad y que asegure la sincronización con la infraestrucutra de datos.
    \item \textbf{O3}\label{o3}. Desarrollar un sistema de procesamiento semántico basado en agentes y grafos de conocimiento que permita la gestión de la información de entrada del usuario en el sistema.
    \item \textbf{04}\label{o4}. Construir una interfaz de usuario que priorice la captura por voz para reducir la fricción de entrada de información por parte del usuario.
    \item \textbf{05}\label{o5}. Implementar mecanismos de seguridad y anonimización para garantizar la privacidad del usuario desde el diseño.
    \item \textbf{06}\label{o6}. Gestionar el ciclo completo de despliegue de la aplicación en la \textit{App Store}, superando los controles de calidad y normativas técnicas exigidas por Apple para la distribución de software comercial.
    \item \textbf{07}\label{o7}. Instrumentar al sistema para la validación de las hipótesis de valor planteadas y para analizar el rendimiento y eficiencia económica del sistema.
\end{itemize}
