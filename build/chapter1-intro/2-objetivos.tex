\section{Objetivos del Proyecto}
\label{s:objetivos}

El objetivo general de este proyecto es diseñar, desarrollar y lanzar un \acrfull{mvp} que sirva como instrumento técnico para validar las hipótesis de valor 
asociadas al propósito de PANOT. Para alcanzar este objetivo general, se han definido los siguientes objetivos específicos:

\begin{itemize}
    \item \textbf{O1}\label{o1}. Investigar y fundamentar el paradigma de la Inteligencia Relacional, analizando las carencias de las soluciones actuales de gestión de contactos para proponer un modelo de datos basado en grafos que capture el contexto de una persona.
    \item \textbf{O2}\label{o2}. Desarrollar un motor de procesamiento semántico basado en agentes de \acrfull{ia}, que sea capaz de automatizar la extracción de entidades y relaciones a partir de lenguaje natural, reduciendo la carga cognitiva del usuario en la fase de registro.
    \item \textbf{O3}\label{o3}. Diseñar una arquitectura de software con enfoque \textit{local-first}, garantizando la soberanía del dato y la disponibilidad total de la herramienta en escenarios de movilidad o ausencia de conectividad.
    \item \textbf{O4}\label{o4}. Implementar una experiencia de usuario \textit{voice-first}, permitiendo demostrar que el uso de este tipo de interfaces de voz es la solución técnica más eficiente para minimizar la fricción en la entrada de datos por parte del usuario.
    \item \textbf{O5}\label{o5}. Instrumentar al sistema para la validación de la viabilidad técnica y económica de un sistema agentico en producción, monitorizando el rendimiento, la latencia y los costes operativos para asegurar que el modelo propuesto es sostenible y escalable.
    \item \textbf{O6}\label{o6}. Garantizar la privacidad y seguridad del usuario desde el diseño (\textit{Privacy and Efficiency by Design}), implementando mecanismos de aislamiento de datos y anonimización de métricas que protejan la información sensible gestionada por el sistema.
    \item \textbf{O7}\label{o7}. Ejecutar el ciclo completo de lanzamiento comercial en la \textit{App Store}, validando el cumplimiento de las normativas de calidad exigidas por Apple.
\end{itemize}

