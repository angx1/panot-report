\section{Motivación y Alcance del Proyecto}
\label{s:motivacion-alcance}

La motivación de este proyecto radica tanto en una necesidad personal como en una profesional. Siempre he tenido problemas para recordar nombres y detalles 
de las personas con las que me cruzo en mi día a día, es por eso que necesitaba de una herramienta que me permitiera sin casi esfuerzo guardar esta información en el teléfono móvil. Las soluciones actuales 
están muy polarizadas, o bien son soluciones muy orientadas al ambito profesional para personas de departamentos de ventas y/o marketing, o bien son soluciones muy estáticas y simples como las aplicaciones de 
contactos que todos tenemos en nuestro teléfono.

El proyecto engloba todo el análisis, diseño, desarrollo y lanzamiento de PANOT. Una aplicación móvil que permite, como se ha comentado, dar una solución más mixta que no solo sirva para guardar números de teléfono 
e información simple de contacto sino que permita gestionar el complejo contexto de una relación interpersonal de manera simple y efectiva. 

Además, este proyecto en su conjunto permitiría validar y documentar un flujo de trabajo que transformase cualquier solución del 0 al 1 \cite{thiel2014zerotoone}, es decir, de una idea a algo tangible, y para 
que cualquier persona que quisiera hacer lo mismo, sirviera como guía o inspiración para hacerlo. 
