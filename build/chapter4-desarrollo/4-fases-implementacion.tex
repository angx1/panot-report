\clearpage
\section{Implementación del Sistema}

Como se mencionó en la sección \ref{sec:enfoque-lean}, el desarrollo de PANOT se centra en la producción de un artefacto que permita 
validar las hipótesis de valor. Es por esto que, a diferencia de un modelo incremental tradicional, la implementación de este sistema 
se ha llevado a cabo como un esfuerzo unificado para materializar el \acrshort{mvp}. Bajo esta premisa, la implementación de PANOT se ha 
organizado en ejes de desarrollo concurrentes para maximizar la velocidad de desarrollo y permitir el trabajo paralelo sobre las diferentes 
capacidades del sistema. 

Es esencial señalar que en este proyecto, el despliegue no es una fase posterior de la implementación, sino su culminación. La publicación 
de la aplicación \textit{"Semilla"} en una plataforma de distribución es el evento de cierre del proceso de construcción (Fase \textit{Construir}) y da comienzo 
a la fase de medición (Fase \textit{Medir}). Por ello, la implementación incluirá tareas como la instrumentación de herramientas de medición ya que, sin ellas, 
el \acrshort{mvp} no podría cumplir su propósito de validación de la hipótesis de valor.

\subsection{Organización \texttt{panot-hq}}

Para facilitar la gestión y organización del código del proyecto, se ha creado la organización \texttt{panot-hq} en \textit{GitHub}. 
Esta organización centraliza todos los repositorios de desarrollo de PANOT, permitiendo una estructura clara y escalable del proyecto. Además, como ya se explicó en 
la sección \ref{sec:metodologia-flujo-desarrollo}, se ha configurado un proyecto dentro de la organización para agrupar las unidades de trabajo y el progreso de las mismas.
Esta estructura se ha establecido para, además de los beneficios obvios de trazabilidad y seguimiento, permitir la incorporación 
de nuevos colaboradores en el futuro. Los repositorios levantados de la organización son los siguientes:

\vspace{.5cm}
\begin{figure}[ht!]
    \centering
    \includegraphics[width=0.9\textwidth]{figures/panot-hq-repos.png}
    \caption{Repositorios de la organización \texttt{panot-hq}}
    \label{fig:panot-hq-repos}
\end{figure}

{\small
\begin{itemize}
    \item \texttt{panot-hq/mobile}: Código de la aplicación móvil.
    \item \texttt{panot-hq/landing}: Código de \gls{landing} de PANOT.
    \item \texttt{panot-hq/edge-backend}: Código de las Edge Functions que se despliegan en Supabase.
    \item \texttt{panot-hq/graph-visualizer}: Código del entorno de pruebas para la visualización de grafos de conocimiento.
    \item \texttt{panot-hq/panot-speech}: Código del paquete \textit{npm} del módulo nativo iOS para transcripción en tiempo real para dispositivos iOS.
\end{itemize}
}

\subsection{Ejes de Desarrollo}

Se define eje de desarrollo como la línea de trabajo dedicada al desarrollo de una capacidad del sistema, incluyendo su integración con el 
resto de los componentes con los que interactúe. En el caso de PANOT, se han llevado a cabo los siguientes ejes de desarrollo:


\begin{table}[ht!]
\footnotesize
\begin{tabularx}{\textwidth}{|l|X|X|}
\hline
\rowcolor[HTML]{EFEFEF} 
\textit{Eje} & \textit{Repositorios Involucrados} & \textit{Tecnologías Utilizadas} \\ \hline
(1) Aplicación Móvil & {\scriptsize \texttt{panot-hq/mobile} \newline \texttt{panot-hq/panot-speech}} & Expo, SwiftUI, TypeScript, PostHog, Stripe \\ \hline
(2) Base de Datos & {\scriptsize \texttt{panot-hq/mobile}} & Supabase, PostgreSQL, TypeScript \\ \hline
(3) Sistema Multi-Agente & {\scriptsize \texttt{panot-hq/edge-backend} \newline \texttt{panot-hq/graph-visualizer} \newline \texttt{panot-hq/mobile}} & Supabase, Deno, TypeScript, LangChain, LangSmith \\ \hline
(4) Sistema de Cola de Trabajos & {\scriptsize \texttt{panot-hq/edge-backend} \newline \texttt{panot-hq/mobile}} & Supabase, Deno, TypeScript \\ \hline
(5) Landing de PANOT & {\scriptsize \texttt{panot-hq/landing}} & Next.js, TypeScript \\ \hline
\end{tabularx}
\caption{Ejes de desarrollo de PANOT}
\label{tab:ejes-trabajo}
\end{table}

{\small
\begin{enumerate}
    \item[(1)] \textbf{Aplicación Móvil}: Desarrollo de la UI y componentes de la aplicación de PANOT.
    \item[(2)] \textbf{Base de Datos}: Desarrollo y despliegue de la infraestructura de datos.
    \item[(3)] \textbf{Sistema Multi-Agente}: Desarrollo de sistema de procesamiento de información.
    \item[(4)] \textbf{Sistema de Cola de Trabajos}: Desarrollo de sistema de gestión de trabajadores.
    \item[(5)] \textbf{\gls{landing} de PANOT}: Desarrollo de la página web.
\end{enumerate}
}

\subsection{Evolución Cronológica}
\label{subsec:evolucion-cronologica}

Para asegurar una evolución estable y controlada del código, se estableció una planificación orientativa centrada en la reducción de incertidumbre técnica 
y basada 4 hitos principales:

{\small
\begin{enumerate}
    \item \textit{Sincronización y Offline-first}: Garantizar que la app fuera funcional sin conexión.
    \item \textit{Captura Nativa}: Validar la transcripción en tiempo real de forma nativa y offline.
    \item \textit{Inteligencia Relacional}: Implementar el sistema multi-agente para el procesamiento de las interacciones y creación de contactos.
    \item \textit{Cierre de Producto}: Implementar sistema de suscripciones y observabilidad para el ciclo "Medir".
\end{enumerate}
}

\vspace{.5cm}
\begin{figure}[ht!]
    \centering
    \includegraphics[width=1\textwidth]{figures/burn-up-lines-stacked.png}
    \caption{Gráfico de burndup de issues cerrados en el proyecto de PANOT}
    \label{fig:timeline-development}
\end{figure}

El desarrollo del \acrshort{mvp} se ha llevado a cabo en un periodo de cuatro meses (octubre 2025 - enero 2026). Se han descartado cambios que se han 
implementado después de estas fechas ya que engloban tanto cambios no críticos como correcciones de errores menores o iteraciones en la interfaz de usuario. 
A continuación, se detalla la línea de tiempo del desarrollo de octubre de 2025 a enero de 2026, siguiendo el orden de prioridad de los hitos enumerados 
anteriormente. 

\subsubsection{Octubre 2025}

{\small
\noindent
\textbf{Ejes de desarrollo:} (1) Aplicación Móvil, (2) Base de Datos. \newline
\textbf{Repositorios:} {\footnotesize \texttt{panot-hq/mobile}}, {\footnotesize \texttt{panot-hq/panot-speech}.} \newline
\textbf{Requisitos Funcionales implementados:} primera parte del \hyperref[req:FR-01]{FR-01}, \hyperref[req:FR-02]{FR-02}, \hyperref[req:FR-04]{FR-04}, \hyperref[req:FR-08]{FR-08}, 
\hyperref[req:FR-10]{FR-10}, \hyperref[req:FR-11]{FR-11}, \hyperref[req:FR-12]{FR-12}, \hyperref[req:FR-13]{FR-13}
}

El objetivo de octubre fue alcanzar una versión estable \textit{offline} con las capacidades mínimas de gestión de usuario, contactos e interacciones. Además, se desarrolló e 
integró el paquete nativo {\footnotesize \texttt{panot-speech}} con la interfaz de usuario, permitiendo al usuario grabar y transcribir su voz en tiempo real y de forma totalmente offline localmente.

\vspace{.5cm}
\begin{figure}[ht!]
    \centering
    \includegraphics[width=0.6\textwidth]{figures/closed-issues-octubre.png}
    \caption{Issues cerradas en octubre de 2025}
    \label{fig:closed-issues-octubre}
\end{figure}

Como se puede ver en la figura \ref{fig:closed-issues-octubre}, se implementó el registro por Email y Apple ID (Issue \#22) y se configuró la capa de persistencia local 
con \textit{Legend-State} (Issue \#33). Adicionalmente, se habilitó parte del \acrfull{crud} básico manual de contactos (Issues \#6, \#7, \#8). Se implementó también la 
interfaz de grabación y la transcripción en tiempo real (Issues \#12, \#26), gestionando los permisos contextuales necesarios y obligatorios de iOS para el uso del micrófono y 
acceso a la agenda de contactos local del usuario (Issue \#25). 

\vspace{.7cm}
\begin{figure}[ht!]
    \centering
    \includegraphics[width=1\textwidth]{figures/git-commit-tree-octubre.png}
    \caption{Árbol de commits de octubre (y finales de septiembre) de 2025 en repositorio Git local de \texttt{panot-hq/mobile}}
    \label{fig:git-commit-tree-octubre}
\end{figure}

\vspace{.7cm}
\subsubsection{Noviembre 2025}
{\small
\noindent
\textbf{Ejes de desarrollo:} (1) Aplicación Móvil, (3) Sistema Multi-Agente (Inicio). \newline
\textbf{Repositorios:} {\footnotesize \texttt{panot-hq/mobile}}. \newline
\textbf{Requisitos Funcionales implementados:} segunda parte del \hyperref[req:FR-01]{FR-01}, \hyperref[req:FR-05]{FR-05}, \hyperref[req:FR-06]{FR-06}, \hyperref[req:FR-07]{FR-07}, \hyperref[req:FR-09]{FR-09}
}

Noviembre se centró en completar las funcionalidades restantes de gestión de contactos además de preparar el sistema para las capacidades de 
inteligencia relacional. El objetivo fue alcanzar un sistema funcional que permitiera la actualización automática de los detalles  
de los contactos mediante el procesamiento de interacciones. Como dato importante, en este punto del proyecto, todavía no se contaba con el sistema 
multi-agente ni la clase de grafo semántico, es decir, todavía no existía un sistema completo de procesamiento relacional ni en el procesamiento de 
interacciones ni en la creación de contactos ”hablando sobre ellos”. Lo implementado fue una solución temporala.

\vspace{.5cm}
\begin{figure}[ht!]
    \centering
    \includegraphics[width=0.7\textwidth]{figures/closed-issues-noviembre.png}
    \caption{Issues cerradas en noviembre de 2025}
    \label{fig:closed-issues-noviembre}
\end{figure}

Como se ve en la Figura \ref{fig:closed-issues-noviembre}, en noviembre se implementó la eliminación de contactos (Issue \#9), su creación mediante lenguaje natural (Issue \#37), la importación desde la 
agenda nativa (Issue \#10) y la búsqueda de contactos (Issue \#11), reduciendo drásticamente la fricción de uso de la aplicación. Por último, se completó el flujo de acceso 
de usuarios (Issue \#23) lo que desbloqueó por completo que el sistema contara con un mecanismo de autenticación robusto y funcional tanto para el \textit{sign-up} (registro) 
como para el \textit{log-in} (inicio de sesión), y se incorporó el apartado de ajustes de la aplicación (Issue \#42), lo que estableció la base para futuras capacidades y el apartado de feedback.

\vspace{.7cm}
\begin{figure}[ht!]
    \centering
    \includegraphics[width=1\textwidth]{figures/git-commit-tree-noviembre.png}
    \caption{Árbol de commits de noviembre de 2025 en repositorio \texttt{panot-hq/mobile} local}
    \label{fig:git-commit-tree-noviembre}
\end{figure}


\newpage
\subsubsection{Diciembre 2025}
{\small
\noindent
\textbf{Ejes de desarrollo:} (1) Aplicación Móvil, (2) Base de Datos, (3) Sistema Multi-Agente, (4) Sistema de Cola de Trabajos. \newline
\textbf{Repositorios:} {\footnotesize \texttt{panot-hq/mobile}}, {\footnotesize \texttt{panot-hq/edge-backend}}, {\footnotesize \texttt{panot-hq/graph-visualizer}}. \newline
\textbf{Requisitos Funcionales implementados:} \hyperref[req:FR-14]{FR-14}, \hyperref[req:FR-15]{FR-15}, \hyperref[req:FR-16]{FR-16}, \hyperref[req:FR-17]{FR-17}
}

Diciembre representó el mes con mayor densidad técnica, y se centró en completar las funcionalidades esenciales para el lanzamiento del \acrshort{mvp}, incluyendo el sistema de suscripciones, 
el sistema multi-agente, el sistema de cola de procesamiento, la capa de observabilidad y los mecanismos de soporte.

\vspace{.5cm}
\begin{figure}[ht!]
    \centering
    \includegraphics[width=0.7\textwidth]{figures/closed-issues-diciembre.png}
    \caption{Issues cerradas en diciembre de 2025}
    \label{fig:closed-issues-diciembre}
\end{figure}

En primer lugar, se completó la lógica del \textit{paywall} y la integración con \textit{Stripe} para suscripciones mensuales 
y acceso a funcionalidades premium (Issue \#50) con el desarrollo de la Edge Function {\footnotesize \texttt{stripe-infra}}. Esta implementación permitió establecer el 
modelo de monetización del sistema y controlar el acceso a las capacidades de procesamiento relacional de la aplicación.

En segundo lugar, se implementó el \hyperref[fig:diagrama-multi-agente]{sistema multi-agente} completo para el procesamiento de información (Issue \#52):
{\small
\begin{enumerate}
    \item Se desarrolló la edge function {\footnotesize \texttt{relational-agent}} en {\footnotesize \texttt{panot-hq/edge-backend}} utilizando \textit{LangChain} 
    para la orquestación de agentes e integración con la aplicación móvil.
    \item Se crearon las tablas necesarias en Supabase para los nodos y las aristas semánticas.
    \item Se desarrolló la herramienta {\footnotesize \texttt{panot-hq/graph-visualizer}} para facilitar el desarrollo, pruebas 
    y validación de la estructura del grafo semántico del usuario.
    \item Y se realizó la integración con \textit{LangSmith} como capa de observabilidad del sistema agéntico, permitiendo así monitorizar el flujo de procesamiento con información como la latencia, los 
    tokens utilizados o el coste de la ejecución de cada llamada al sistema \ref{fig:ejemplo-traza-langsmith}.
\end{enumerate}
}

\begin{figure}[ht!]
    \centering
    \includegraphics[width=1\textwidth]{figures/ejemplo-traza-langsmith.png}
    \caption{Ejemplo de traza de ejecución en LangSmith del sistema multi-agentede PANOT}
    \label{fig:ejemplo-traza-langsmith}
\end{figure}
\vspace{.5cm}

Para la gestión de la ejecución asíncrona de procesos entre usuarios, se implementó el sistema de cola de procesos, 
como puede verse en la Figura~\ref{fig:processing-system-activity-diagram}. Entre los principales sub-hitos de esta implementación se encuentran 
el desarrollo de {\footnotesize \texttt{worker-infra}} en {\footnotesize \texttt{panot-hq/edge-backend}}, su integración con la aplicación móvil, 
así como la creación de las tablas necesarias en Supabase tanto para la cola de procesos como para los \textit{workers} de los usuarios.

Por último, se instrumentó la aplicación para el análisis de comportamiento de usuarios utilizando \textit{PostHog} (Epic \#55). Esta implementación permitió la 
captura de métricas para validar las hipótesis de valor definidas en el proyecto. Y se añadió el mecanismo de soporte y recolección de \textit{tickets} para reportes 
de errores o solicitudes de funcionalidades desde la aplicación (Issue \#43) lo que instauró el canal de comunicación entre 
los usuarios y el equipo de desarrollo (por ahora unipersonal).

\vspace{.7cm}
\begin{figure}[ht!]
    \centering
    \includegraphics[width=1\textwidth]{figures/git-commit-tree-diciembre.png}
    \caption{Árbol de commits de diciembre de 2025 (y principios de enero de 2026) en repositorio Git local de \texttt{panot-hq/mobile}}
    \label{fig:git-commit-tree-diciembre}
\end{figure}

\vspace{.7cm}
\subsubsection{Enero 2026}

{\small
\noindent
\textbf{Ejes de desarrollo:} (1) Aplicación Móvil (Cierre), (5) \gls{landing} de PANOT. \newline
\textbf{Repositorios:} {\footnotesize \texttt{panot-hq/mobile}}, {\footnotesize \texttt{panot-hq/landing}}. \newline
\textbf{Requisitos Funcionales implementados:} \hyperref[req:FR-03]{FR-03}
}

Habiendo finalizado el desarrollo completo del \acrshort{mvp}, principios de enero (del 6 al 8) de 2026 se dedicaron al desarrollo de la 
\gls{landing} de PANOT tanto en inglés como en español además de incorporar la sección de gestión de cuenta en la aplicación, donde el ususario 
puede gestionar su suscripción o eliminar permanentemente su cuenta.

\vspace{.5cm}
\begin{figure}[ht!]
    \centering
    \includegraphics[width=0.55\textwidth]{figures/closed-issues-enero.png}
    \caption{Issue cerrada en enero de 2026 (captura tomada a mediados de enero de 2026)}
    \label{fig:closed-issues-enero}
\end{figure}





