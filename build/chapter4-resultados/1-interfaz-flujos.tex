
\section{Interfaz de Usuario y Flujos del Sistema}

El diseño de la interfaz de PANOT se ha construido, como ya se ha comentado, siguiendo las \acrshort{hig} de Apple, apoyándose
sobre los tres pilares fundamentales de: \textit{Jerarquía}, \textit{Armonía} y \textit{Consistencia}. La \textit{Jerarquía} visual permite que el 
usuario identifique instantáneamente las acciones críticas, como la captura de voz; la \textit{Armonía} asegura una paleta de colores y tipografías que reducen la 
fatiga visual; y la \textit{Consistencia} garantiza que los patrones de navegación (como la barra de navegación o los botones de ajustes, añadir o aceptar) sean 
familiares para cualquier usuario de la plataforma.

A continuación, se ilustran los flujos de la aplicación siguiendo las Historias de Usuario Principales:

\subsection{Flujo de Registro de Usuario y Onboarding}

[incluir capturas]

\subsection{Flujo de Inicio de Sesión}

[incluir capturas]

\subsection{Flujo de Creación de Contacto}

[incluir capturas]

\subsection{Flujo de Grabación de Interacción y Asignación a Contacto}

[incluir capturas]

\subsection{Flujo de Visualización de Interacciones de un Contacto}

[incluir capturas]

\subsection{Flujo de Configuración y Ajustes}

[incluir capturas]

\subsection{Flujo de Soporte y Feedback}

[incluir capturas]