%%%%%%%%%%%%%%%%%%%%%%%%%%%%%%%%%%%
% ANÁLISIS COMPARATIVO

\section{Contexto tecnológico}


\subsection{Arquitectura de PANOT}

Para materializar estos valores, el sistema se ha estructurado en dos componentes arquitectónicos fundamentales que operan de manera complementaria:

\begin{itemize}
    \item \textbf{PANOT (cliente)}: Constituye la capa de contacto entre el usuario y el sistema, proporcionando una 
    interfaz intuitiva y adaptativa que facilita la captura de interacciones, la visualización de sus relaciones, y la 
    presentación de recomendaciones contextuales entre otras características. 
    El diseño de esta aplicación se centra en maximizar la conexión emocional con el usuario y en proporcionar 
    una experiencia orgánica y sin fricción. 
    
    \item \textbf{PANOT OS (servidor)}: Representa la capa de inteligencia o el cerebro 
    del sistema, encargada de procesar, analizar y aprender continuamente de las interacciones capturadas por la 
    aplicación móvil. PANOT OS es el responsable de construir y mantener las representaciones relacionales estructuradas, 
    de inferir patrones y preferencias del usuario, de generar recomendaciones contextuales y de evolucionar 
    dinámicamente el conocimiento sobre las relaciones. 
\end{itemize}

En este marco, PANOT se encarga de crear la conexión emocional y facilitar la interacción inmediata, 
mientras que PANOT OS proporciona la personalización adaptativa y la comprensión contextual que permiten que esa 
conexión se profundice con el tiempo. Ambos componentes trabajan de manera sinérgica para materializar los valores 
diferenciadores de conexión emocional y personalización adaptativa, garantizando que el sistema no solo responda a las 
necesidades actuales del usuario, sino que evolucione continuamente para acompañar la dinámica natural de sus relaciones.
