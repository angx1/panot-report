\section{Cambio de Paradigma en el Diseño de Producto}

Un paradigma de diseño de producto constituye un conjunto de principios fundamentales, patrones de interacción y enfoques 
conceptuales que guían la creación de experiencias de usuario en productos y servicios digitales o analógicos. Representa más que una 
simple metodología de diseño; es una filosofía que establece cómo los usuarios perciben, interactúan y se relacionan 
con una aplicación. Fijese que este apartado no está orientado en principios de diseño de la arquitectura del software, sino en 
principios de diseño de producto que van más allá del desarrollo de la aplicación y están orientados en la experiencia del usuario.

En el contexto de las aplicaciones móviles, la relevancia de los paradigmas de diseño de producto adquiere una dimensión crítica 
debido a las características inherentes de estos dispositivos: limitaciones de espacio en pantalla, interacciones 
predominantemente táctiles y expectativas de inmediatez y estímulo por parte de los usuarios. 
Un paradigma de diseño adecuado no solo determina la usabilidad de una aplicación, sino que establece la base sobre la 
cual se construyen las expectativas del usuario, su curva de aprendizaje y, fundamentalmente, su conexión emocional con el producto.

La irrupción de la Inteligencia Artificial como tecnología dominante ha transformado radicalmente el panorama del diseño de productos digitales. 
La democratización de las capacidades de procesamiento de los modelos de lenguaje ha generado un desplazamiento del valor diferencial de los productos: ya no es suficiente ofrecer una funcionalidad 
única, pues estas características pueden replicarse rápidamente. En este nuevo contexto, la diferenciación competitiva 
se desplaza hacia dimensiones más profundas y fundamentales de la experiencia humana.

Es decir, los usuarios no solo buscan que una aplicación funcione bien; buscan que se \textit{adapte} a ellos, 
que \textit{comprenda} su contexto, que \textit{evolucione} con sus necesidades y que establezca una conexión 
que trascienda la mera transacción funcional.

Con esta premisa, PANOT se diseña con una orientación centrada en el usuario y su contexto, buscando una experiencia adaptada a sus necesidades. 
Aunque en su versión actual no despliegue una experiencia de interfaz única para cada usuario ni perfiles de personalización explícitos, se alinea 
con los valores diferenciadores anteriores en dos frentes: en \textit{conexión y resonancia emocional}, al posicionar la aplicación como extensión 
cognitiva de las relaciones del usuario ya que el sistema acompañaría la evolución de cada contacto del usuario y su contexto relacional, y en \textit{comprensión contextual}, 
al inferir de las interacciones capturadas del usuario patrones, intereses y necesidades de sus contactos sin exigirle configuraciones manuales. 

