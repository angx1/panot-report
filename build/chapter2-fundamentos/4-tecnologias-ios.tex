%%%%%%%%%%%%%%%%%%%%%%%%%%%%%%%%%%%
% TECNOLOGÍAS PARA EL DESARROLLO DE APLICACIONES iOS

\section{Tecnologías para el Desarrollo de Aplicaciones para iOS}

\subsection{Desarrollo Nativo}

El desarrollo nativo para iOS consiste en crear aplicaciones utilizando específicamente las herramientas, 
lenguajes y frameworks proporcionados por Apple para la plataforma. Este enfoque permite aprovechar al máximo 
las capacidades del sistema operativo iOS y de los dispositivos de Apple, garantizando un rendimiento óptimo y 
acceso completo a todas las funcionalidades del sistema.

El lenguaje de programación predominante en el desarrollo nativo para iOS es \textit{Swift}, introducido por 
Apple en 2014 como sucesor de \textit{Objective-C}. Swift es un lenguaje moderno, seguro y de alto rendimiento 
que combina características de programación orientada a objetos y funcional, diseñado específicamente para ser 
más intuitivo y menos propenso a errores que su predecesor. Aunque \textit{Objective-C} se sigue manteniendo en proyectos 
legados o de alto riesgo de migración.

La herramienta principal para el desarrollo nativo es \textit{Xcode}, el entorno de desarrollo integrado (IDE) 
oficial de Apple. Xcode proporciona un editor de código, compilador, depurador, simulador de iOS, herramientas 
de gestión de interfaces y un sistema completo de gestión de proyectos. Además, incluye \textit{Interface Builder} 
para diseñar interfaces de usuario de forma visual e \textit{Instruments} para análisis de rendimiento y detección 
de fugas de memoria.

Para la construcción de interfaces de usuario, Apple proporciona dos frameworks principales: \textit{UIKit}, el 
framework tradicional basado en programación imperativa y eventos, y \textit{SwiftUI}, introducido en 2019, que 
utiliza un paradigma declarativo y permite crear interfaces de manera más moderna y eficiente. SwiftUI facilita 
el desarrollo de interfaces adaptativas y reactivas, aunque UIKit sigue siendo ampliamente utilizado, especialmente 
en proyectos existentes o cuando se requiere mayor control sobre el comportamiento de la interfaz. El desarrollo 
nativo también permite el uso directo de \textit{Cocoa Touch}, la capa de frameworks de iOS que incluye servicios 
fundamentales como \textit{Core Data} para persistencia de datos, \textit{Core ML} para modelos de lenguaje, 
\textit{Core Location} para servicios de ubicación, y acceso completo a las APIs del sistema operativo.

\subsection{Frameworks Multiplataforma}

Además del desarrollo nativo, existen varios frameworks multiplataforma que permiten desarrollar aplicaciones 
para iOS junto con otras plataformas (principalmente Android) desde una base de código compartida.
{\small
\begin{itemize}
    \item \textit{React Native}, desarrollado por Meta, permite crear aplicaciones móviles utilizando JavaScript y React. 
    El framework utiliza un puente nativo que comunica el código JavaScript con componentes nativos de cada plataforma, 
    permitiendo acceso a APIs nativas mientras se comparte la mayor parte de la lógica de negocio.
    
    \item \textit{Expo}, construido sobre React Native, proporciona un conjunto de herramientas y servicios que simplifican 
    el desarrollo, incluyendo un runtime unificado, APIs listas para usar y un sistema de compilación en la nube. Expo 
    reduce significativamente la complejidad de configuración del proyecto y facilita el despliegue, aunque con algunas 
    limitaciones en el acceso a funcionalidades nativas avanzadas.
    
    \item \textit{Flutter}, desarrollado por Google, utiliza el lenguaje \textit{Dart} y un motor de renderizado propio que 
    compila a código nativo. Flutter construye la interfaz de usuario desde cero en cada plataforma, evitando la 
    necesidad de componentes nativos del sistema operativo y proporcionando mayor consistencia visual entre plataformas.
\end{itemize}
}
Otras alternativas multiplataforma incluyen \textit{Xamarin} (ahora \textit{.NET MAUI}) que utiliza C\# y .NET, 
\textit{Ionic} que combina tecnologías web (HTML, CSS, JavaScript) con capacidades nativas mediante \textit{Cordova}, 
y \textit{Unity} para aplicaciones que requieren capacidades gráficas avanzadas, esencialmente, videojuegos.

La elección entre desarrollo nativo y multiplataforma depende de factores como requisitos de rendimiento, necesidad 
de acceso a funcionalidades nativas avanzadas, tiempo de desarrollo, mantenimiento a largo plazo o recursos del equipo.

\subsection{Proceso de Desarrollo y Distribución}

El ciclo de vida de una aplicación iOS desde el desarrollo hasta su distribución sigue un proceso estructurado:

\begin{enumerate}
    \item \textit{Desarrollo}: Durante esta fase, el código se compila y ejecuta en simuladores iOS o dispositivos físicos mediante 
    perfiles de desarrollo.
    
    \item \textit{Pruebas internas}: Para pruebas internas, las aplicaciones, una vez compiladas, es posible su distribución mediante 
    \textit{TestFlight}, plataforma que permite a desarrolladores invitar hasta 10.000 beta testers externos sin 
    necesidad de certificados adicionales.
    
    \item \textit{Archive}: El proceso de \textit{Archive} genera un paquete de distribución de la aplicación (\textit{.ipa}) optimizado 
    y firmado con los certificados de distribución \footnote{Los certificados de distribución son credenciales digitales emitidas por 
    Apple que permiten identificar y autenticar al desarrollador, garantizando que la aplicación proviene de una fuente confiable y 
    verificada. Son esenciales para firmar digitalmente las aplicaciones antes de su distribución en la App Store.}.
    Esta versión archivada puede subirse a \textit{App Store Connect}, portal web de Apple para gestión de aplicaciones, donde se 
    configura información de la aplicación, capturas de pantalla, descripciones y metadatos requeridos para la publicación.
    
    \item \textit{App Review}: Una vez hecha la petición de subida, este proceso de revisión de Apple verifica que el producto cumple con 
    las directrices de la App Store, incluyendo seguridad, privacidad, calidad técnica y contenido. Una vez aprobada, la aplicación 
    está disponible para distribución pública o privada según la configuración establecida.
    
    \item \textit{Distribución}: La distribución puede realizarse mediante tres canales principales:
    {\small
    \begin{itemize}
        \item \textbf{App Store}: Para usuarios finales a través de la tienda oficial de Apple.
        \item \textbf{Distribución empresarial (\textit{Enterprise})}: Permite a las organizaciones internas distribuir aplicaciones privadas a sus empleados, fuera de la App Store.
        \item \textbf{Distribución ad-hoc}: Permite instalar la aplicación en un número limitado de dispositivos específicos, identificados mediante perfiles de aprovisionamiento
        \footnote{Los perfiles de aprovisionamiento son archivos que vinculan a un desarrollador y su aplicación con una cuenta de desarrollador, dispositivos y servicios autorizados}.
    \end{itemize}
    }
\end{enumerate}
