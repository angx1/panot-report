
\section{Gestión de la Inteligencia Relacional}

Los sistemas CRM convencionales, si bien útiles en entornos corporativos para la gestión masiva de clientes, presentan 
limitaciones significativas cuando se trata de capturar la complejidad inherente a las relaciones humanas. Estos sistemas 
operan principalmente con información descontextualizada, almacenando datos de contacto de manera estática y registrando 
interacciones sin considerar su evolución temporal ni el contexto en el que ocurren.

La Inteligencia Relacional presenta un nuevo paradigma evolutivo en la gestión de información personal y profesional, introduciendo el
concepto de dinamismo contextual, donde cada relación evoluciona continuamente reflejando cambios en intereses, preferencias y circunstancias vitales. 
Este enfoque reconoce que las relaciones que tenemos no son entidades fijas, sino procesos complejos que cambian según un contexto temporal y situacional.


\subsection{Inteligencia Relacional en el Contexto de la Inteligencia Artificial}

Para comprender el alcance de la Inteligencia Relacional, es necesario enmarcar el concepto dentro del ecosistema más amplio
de la Inteligencia Artificial y analizar cómo se diferencia con los paradigmas tradicionales.

La diferencia fundamental entre la Inteligencia Relacional y los paradigmas tradicionales de IA radica en que, mientras estos últimos 
operan mediante asociaciones estadísticas entre patrones de entrada y salida —optimizando funciones de pérdida sobre grandes volúmenes 
de datos descontextualizados—, la Inteligencia Relacional se fundamenta en la construcción y manipulación de representaciones 
estructurales de relaciones que permiten la generalización cruzada y la inferencia analógica\footnote{Transmisión de conocimientos de una situación a otra}. 
La Inteligencia Relacional captura la estructura relacional subyacente que puede  transferirse entre dominios aparentemente no relacionados, 
tal como ocurre en el razonamiento humano.

La investigación en Inteligencia Relacional demuestra capacidades que van más allá del aprendizaje estadístico tradicional. 
Doumas et al. \cite{doumas2022theory} muestran cómo un modelo computacional puede aprender representaciones relacionales 
estructuradas y realizar generalización de cero disparos\footnote{Escenario de aprendizaje automático en el que se entrena un modelo de IA para reconocer y categorizar 
objetos o conceptos sin haber visto ningún ejemplo de esas categorías o conceptos de antemano} entre dominios completamente diferentes, como la transferencia de 
conocimiento entre videojuegos. Esta capacidad de generalización permite que el sistema aprenda a reconocer y comprender relaciones 
entre entidades en contextos completamente diferentes.

Traspasando la analogía de los videojuegos al contexto de PANOT, la Inteligencia Relacional como se menciona en \cite{doumas2022theory} permite que el sistema
aprenda a reconocer y comprender patrones relacionales estructurados entre personas, eventos y contextos, más allá de las asociaciones estadísticas superficiales. 
En lugar de simplemente almacenar datos estáticos de contactos, PANOT puede construir representaciones relacionales dinámicas que capturan la estructura subyacente 
de las relaciones humanas —como la evolución temporal de intereses comunes, la frecuencia contextual de interacciones, o los cambios en preferencias y circunstancias vitales—. 

Por ejemplo, el sistema puede reconocer que ciertos patrones de comunicación efectivos en relaciones profesionales pueden generalizarse a nuevas relaciones profesionales, 
o que cambios detectados en el contexto de una relación personal pueden aplicarse para comprender dinámicas similares en otras relaciones. Esta generalización relacional 
es lo que permite que PANOT evolucione continuamente cada contacto, reflejando no solo quién es esa persona en términos estáticos, sino cómo ha evolucionado y continúa 
evolucionando la relación según el contexto temporal y situacional. Para ilustrar este proceso, consideremos un ejemplo práctico del flujo de procesamiento relacional en PANOT:

\underline{Input:} El usuario captura una interacción mediante nota de voz: ``Acabo de almorzar con María. Está muy emocionada 
porque ha conseguido un nuevo trabajo como diseñadora en una startup tecnológica. Le interesa especialmente el trabajo remoto 
y mencionó que está buscando un piso más cerca de su nueva oficina. Hablamos de proyectos de diseño colaborativo y se mostró 
muy receptiva a la idea de futuros proyectos juntos.''

\underline{Procesamiento:} PANOT procesa esta entrada multimodal extrayendo múltiples capas de información relacional 
estructurada:

{\small
\begin{itemize}
    \setlength{\itemsep}{0pt}
    \setlength{\parsep}{0pt}
    \setlength{\topsep}{0pt}
    \setlength{\partopsep}{0pt}
    \item \textit{Evento}: almuerzo social de contexto informal
    \item \textit{Cambio de estado}: transición profesional — nuevo trabajo como diseñadora
    \item \textit{Cambio de preferencias}: prioridad hacia trabajo remoto
    \item \textit{Necesidad emergente}: búsqueda de vivienda
    \item \textit{Relaciones}: interés común en proyectos de diseño colaborativo, receptividad a futura colaboración
    \item \textit{Contexto temporal}: estado emocional positivo, momento de transición vital
\end{itemize}
} 

El sistema construye una representación relacional estructurada que conecta estas entidades (usuario-contacto)
mediante relaciones semánticas representadas en formato \texttt{JSON}:

\begin{lstlisting}[
    basicstyle=\ttfamily\small,
    frame=single,
    breaklines=true,
    breakatwhitespace=false,
    keepspaces=true,
    showspaces=false,
    showstringspaces=false,
    tabsize=2
]
{
  "ha-cambiado-preferencia": {
    "preferencia": "trabajo-remoto"
  },
  "interés-común": {
    "usuarios": ["Usuario", "María"],
    "tema": ["diseño", "startup-tecnológica", diseño-colaborativo]
  },
  "contexto-temporal-situacional": {
    "evento": "almuerzo-informal",
    "estado": "transición-profesional"
  }
}
\end{lstlisting}

\underline{Output:} PANOT actualiza dinámicamente el contacto de María, generando múltiples outputs contextuales:

{\small
\begin{itemize}
    \setlength{\itemsep}{0pt}
    \setlength{\parsep}{0pt}
    \setlength{\topsep}{0pt}
    \setlength{\partopsep}{0pt}
    \item \textit{Actualización automática del perfil}: se añade ``Diseñadora en startup tecnológica'' como situación laboral actual y se marca ``Trabajo remoto'' como preferencia. Se registra también el cambio de estado como una nueva etapa profesional.
    \item \textit{Recordatorio contextual}: en la ficha de María se crea un recordatorio automático para preguntar sobre la búsqueda de piso en futuras interacciones.
    \item \textit{Recomendaciones de conversación}: el sistema sugiere abordar temas de ``diseño colaborativo'' y ``startups tecnológicas'' en próximos contactos, reforzando el interés común detectado en el \texttt{JSON}.
    \item \textit{Inferencia relacional}: mediante patrones previos, el sistema detecta que los cambios laborales suelen ir acompañados de mayor apertura a colaboraciones y recomienda estrategias de seguimiento específicas para situaciones de transición profesional.
    \item \textit{Seguimiento temporal}: clasifica la interacción dentro de una ``fase de transición profesional positiva'', vinculándola en el timeline relacional y ajustando las siguientes recomendaciones conforme evolucione el contexto.
\end{itemize}
}

