%%%%%%%%%%%%%%%%%%%%%%%%%%%%%%%%%%%
% PRINCIPIOS DE PRIVACIDAD Y EFICIENCIA POR DISEÑO

\section{Principios de Privacidad y Eficiencia por Diseño}

\subsection{Concepto y Fundamentos}

[Explicar qué significa Privacy by Design y Efficiency by Design como principios fundamentales. Hablar sobre su origen, importancia en el contexto actual de aplicaciones que procesan datos personales, y cómo se han convertido en requisitos esenciales tanto desde una perspectiva regulatoria (GDPR, etc.) como desde una perspectiva de diseño de productos que buscan la confianza del usuario.]

\subsection{Aplicación en Sistemas con Modelos de Lenguaje}

[Explicar cómo estos principios se aplican específicamente en aplicaciones que utilizan modelos de lenguaje (LLMs). Cubrir aspectos como:
- Privacidad: minimización de datos, procesamiento local cuando es posible, encriptación de datos sensibles, control granular sobre qué datos se comparten con servicios externos, transparencia sobre qué datos se procesan y cómo
- Eficiencia: optimización de llamadas a APIs, uso de modelos más pequeños y eficientes cuando es posible, caching inteligente de respuestas, procesamiento asíncrono, reducción de latencia, optimización de costos computacionales]

\subsection{Implementación en PANOT}

[Explicar cómo PANOT implementa estos principios en su arquitectura y diseño:
- Privacidad: cómo se manejan los datos personales y relaciones del usuario, qué información se procesa localmente vs. en el servidor, medidas de seguridad implementadas, control del usuario sobre sus datos
- Eficiencia: cómo se optimiza el uso de modelos de lenguaje en PANOT OS, estrategias para reducir llamadas innecesarias a APIs, gestión eficiente de recursos, optimización de la experiencia de usuario minimizando esperas]

