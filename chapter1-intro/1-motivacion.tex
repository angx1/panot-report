\section{Motivación}
\label{s:motivacion}

La idea de este proyecto radica de una necesidad común tanto en el ámbito personal como en el profesional: la limitación que tiene el cerebro de retener detalles o contextos específicos de las 
numerosas interacciones sociales que nos ocurren en nuestro día a día. Las soluciones actuales para satisfacer esta necesidad están muy polarizadas, o bien son soluciones muy orientadas al 
ambito profesional para personal de departamentos de ventas u otros, o bien son soluciones muy estáticas y simples como las aplicaciones de contactos en nuestros teléfonos móviles que no 
permiten capturar la evolución y el matiz de una relación interpersonal.

PANOT nace con el propósito de ocupar este hueco mediante el concepto de Inteligencia Relacional. El proyecto abarca el ciclo completo de ingeniería de software, desde el análisis de requisitos y el diseño 
de su arquitectura hasta el desarrollo y lanzamiento comercial de una aplicación móvil. Además de la solución técnica, se pretende que esta memoria en su conjunto permitiría validar y documentar un 
flujo de trabajo que transformase cualquier solución del 0 al 1 \cite{thiel2014zerotoone}, es decir, de una idea a algo tangible, y para que cualquier persona que quiera hacer lo mismo, sirva 
como guía o inspiración para hacerlo. 